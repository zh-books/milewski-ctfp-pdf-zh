% !TEX root = ../../ctfp-print.tex

\lettrine[lhang=0.17]{如}{果我们将}自函子(endofunctors)解释为定义表达式的方法,那么代数(algebras)允许我们对它们进行求值,而 Monad 则允许我们构造和操作它们。通过将代数与 Monad 结合,我们不仅获得了许多功能,还可以解答一些有趣的问题。

其中一个问题涉及 Monad 和伴随(adjunctions)之间的关系。正如我们所见,每个伴随关系都会\hyperref[monads-categorically]{定义一个 Monad}(以及一个 comonad)。问题是:是否每个 Monad(或 comonad)都可以从伴随关系中导出?答案是肯定的。有一整组伴随关系可以生成给定的 Monad。我将向你展示两种这样的伴随关系。

\begin{figure}[H]
  \centering
  \includegraphics[width=0.25\textwidth]{images/pigalg.png}
\end{figure}

\noindent
让我们回顾一下定义。Monad 是一个自函子 $m$,配备了两个满足某些一致性条件的自然变换(natural transformations)。这些变换在 $a$ 处的分量是:
\begin{align*}
  \eta_a & \Colon a \to m\ a         \\
  \mu_a  & \Colon m\ (m\ a) \to m\ a
\end{align*}
同一个自函子的代数是选择一个特定对象——载体 $a$,以及一个态射:
\[\mathit{alg} \Colon m\ a \to a\]
首先要注意的是,代数的方向与 $\eta_a$ 相反。直观上讲,$\eta_a$ 从类型为 $a$ 的值创建一个平凡的表达式。使代数与 Monad 兼容的第一个一致性条件确保使用载体为 $a$ 的代数对该表达式进行求值时,我们会得到原始值:
\[\mathit{alg} \circ \eta_a = \id_a\]
第二个条件来自于这样一个事实,即有两种方法可以对双重嵌套的表达式 $m\ (m\ a)$ 进行求值。我们可以先应用 $\mu_a$ 来展平表达式,然后使用代数的求值器;或者我们可以先应用提升的求值器来求值内部表达式,然后将求值器应用于结果。我们希望这两种策略是等价的:
\[\mathit{alg} \circ \mu_a = \mathit{alg} \circ m\ \mathit{alg}\]
这里,\code{m alg} 是通过使用函子 $m$ 提升 $\mathit{alg}$ 而得到的态射。以下交换图描述了这两个条件(我将 $m$ 替换为 $T$,以预示接下来的内容):

\begin{figure}[H]
  \centering
  \begin{subfigure}
    \centering
    \begin{tikzcd}[column sep=large, row sep=large]
      a \arrow[rd, equal] \arrow[r, "\eta_a"]
      & Ta \arrow[d, "\mathit{alg}"] \\
      & a
    \end{tikzcd}
  \end{subfigure}
  \hspace{1cm}
  \begin{subfigure}
    \centering
    \begin{tikzcd}[column sep=large, row sep=large]
      T(Ta) \arrow[r, "T\ \mathit{alg}"] \arrow[d, "\mu_a"]
      & Ta \arrow[d, "\mathit{alg}"] \\
      Ta \arrow[r, "\mathit{alg}"]
      & a
    \end{tikzcd}
  \end{subfigure}
\end{figure}

\noindent
我们也可以在 Haskell 中表达这些条件:

\src{snippet01}
让我们看一个小例子。列表自函子的代数包括某个类型 \code{a} 和一个从 \code{a} 列表生成 \code{a} 的函数。我们可以通过选择元素类型和累加器类型都为 \code{a},使用 \code{foldr} 来表达这个函数:

\src{snippet02}
这个特定的代数由一个两参数函数(我们称之为 \code{f})和一个值 \code{z} 来指定。列表函子恰好也是一个 Monad,其中 \code{return} 将值转化为一个单元素列表。代数(这里是 \code{foldr f z})在 \code{return} 之后的组合将 \code{x} 转换为:

\src{snippet03}
这里 \code{f} 的操作以中缀表示。代数与 Monad 兼容的条件是,对于每个 \code{x},以下一致性条件必须得到满足:

\src{snippet04}
如果我们将 \code{f} 视为一个二元操作符,这个条件告诉我们 \code{z} 是右单位元。

第二个一致性条件作用于列表的列表。 \code{join} 的作用是将各个列表连接起来。然后我们可以折叠(fold)结果列表。另一种方式是先折叠各个列表,然后折叠结果列表。同样,如果我们将 \code{f} 解释为一个二元操作符,这个条件告诉我们这个二元操作是结合律的。当 \code{(a, f, z)} 是一个 Monoid 时,这些条件肯定得到了满足。

\section{T-代数(T-algebras)}

由于数学家更喜欢将他们的 Monad 称为 $T$,他们将与之兼容的代数称为 T-代数。给定 Monad $T$ 的 T-代数在范畴 $\cat{C}$ 中形成一个范畴,称为 Eilenberg-Moore 范畴,通常表示为 $\cat{C}^T$。该范畴中的态射是代数的同态。这些与我们之前为 F-代数定义的同态是一样的。

T-代数是一个载体对象和一个求值器(即对的 $(a, f)$)。从 $\cat{C}^T$ 到 $\cat{C}$ 存在一个明显的遗忘函子(forgetful functor)$U^T$,它将 $(a, f)$ 映射到 $a$。它还将 T-代数同态映射到 $\cat{C}$ 中载体对象之间的相应态射。你可能还记得我们讨论伴随关系时提到过,遗忘函子的左伴随称为自由函子(free functor)。

$U^T$ 的左伴随称为 $F^T$。它将范畴 $\cat{C}$ 中的一个对象 $a$ 映射到 $\cat{C}^T$ 中的一个自由代数。这个自由代数的载体是 $T a$。它的求值器是从 $T (T a)$ 返回到 $T a$ 的态射。由于 $T$ 是一个 Monad,我们可以使用 Monad 的 $\mu_a$(在 Haskell 中是 \code{join})作为求值器。

我们仍然需要证明这是一个 T-代数。为此,必须满足两个一致性条件:
\begin{align*}
  \mathit{alg} & \circ \eta_{Ta} = \id_{Ta}     \\
  \mathit{alg} & \circ \mu_a = \mathit{alg} \circ T\ \mathit{alg}
\end{align*}
但如果将 $\mu$ 插入代数,这些条件就是 Monad 的定律。

你可能还记得,每个伴随关系都会定义一个 Monad。事实证明,$F^T$ 和 $U^T$ 之间的伴随关系定义了用于构造 Eilenberg-Moore 范畴的 Monad $T$。由于我们可以对每个 Monad 执行此构造,因此我们得出结论:每个 Monad 都可以从一个伴随关系中生成。稍后我将向你展示另一种生成相同 Monad 的伴随关系。

这是计划:首先我将向你展示 $F^T$ 确实是 $U^T$ 的左伴随。我将通过定义这个伴随的单元(unit)和余单元(counit),并证明相应的三角恒等式得以满足。然后我将向你展示由此伴随关系生成的 Monad 确实是我们原来的 Monad。

伴随的单元是自然变换:
\[\eta \Colon I \to U^T \circ F^T\]
让我们计算这个变换的 $a$ 分量。恒等函子给我们 $a$。自由函子产生自由代数 $(T a, \mu_a)$,而遗忘函子将其简化为 $T a$。总之,我们得到一个从 $a$ 到 $T a$ 的映射。我们将简单地使用 Monad $T$ 的单元作为此伴随的单元。

让我们看看余单元:
\[\varepsilon \Colon F^T \circ U^T \to I\]
让我们计算其在某个 T-代数 $(a, f)$ 处的分量。遗忘函子遗忘了 $f$,自由函子生成对 $(T a, \mu_a)$。因此,为了定义余单元 $\varepsilon$ 在 $(a, f)$ 处的分量,我们需要在 Eilenberg-Moore 范畴中合适的态射,或者是 T-代数的同态:
\[(T a, \mu_a) \to (a, f)\]
这种同态应该将载体 $T a$ 映射到 $a$。让我们复活被遗忘的求值器 $f$。这次我们将其用作 T-代数的同态。事实上,使 $f$ 成为 T-代数的同一交换图可以重新解释为表明它是 T-代数的同态:

\begin{figure}[H]
  \centering
  \begin{tikzcd}[column sep=large, row sep=large]
    T(Ta) \arrow[r, "T f"] \arrow[d, "\mu_a"]
    & Ta \arrow[d, "f"] \\
    Ta \arrow[r, "f"]
    & a
  \end{tikzcd}
\end{figure}

\noindent
因此,我们定义了自然变换 $\varepsilon$ 在 $(a, f)$(T-代数范畴中的一个对象)处的分量为 $f$。

要完成这个伴随关系,我们还需要证明单元和余单元满足三角恒等式。这些是:

\begin{figure}[H]
  \centering
  \begin{subfigure}
    \centering
    \begin{tikzcd}[column sep=large, row sep=large]
      Ta \arrow[rd, equal] \arrow[r, "T \eta_a"]
      & T(Ta) \arrow[d, "\mu_a"] \\
      & Ta
    \end{tikzcd}
  \end{subfigure}%
  \hspace{1cm}
  \begin{subfigure}
    \centering
    \begin{tikzcd}[column sep=large, row sep=large]
      a \arrow[rd, equal] \arrow[r, "\eta_a"]
      & Ta \arrow[d, "f"] \\
      & a
    \end{tikzcd}
  \end{subfigure}
\end{figure}

\noindent
第一个恒等式成立是因为 Monad $T$ 的单位律。第二个恒等式只是 T-代数 $(a, f)$ 的定律。

我们已经证明了这两个函子构成了一个伴随关系:
\[F^T \dashv U^T\]
每个伴随关系都会产生一个 Monad。环形往返
\[U^T \circ F^T\]
是范畴 $\cat{C}$ 中产生相应 Monad 的自函子。让我们看看它对对象 $a$ 的作用。由 $F^T$ 创建的自由代数是 $(T a, \mu_a)$。遗忘函子 $U^T$ 丢弃求值器。因此,确实如此:
\[U^T \circ F^T = T\]
正如预期的那样,此伴随关系的单元是 Monad $T$ 的单元。

你可能还记得,伴随关系的余单元通过以下公式产生 Monad 乘法:
\[\mu = R \circ \varepsilon \circ L\]
这是三个自然变换的水平组合,其中两个是将 $L$ 映射到 $L$ 和将 $R$ 映射到 $R$ 的恒等自然变换。中间的那个,余单元,是一个自然变换,其在代数 $(a, f)$ 处的分量是 $f$。

让我们计算 $\mu_a$ 的分量。我们首先在 $F^T$ 之后水平组合 $\varepsilon$,结果是 $\varepsilon$ 在 $F^T a$ 处的分量。由于 $F^T$ 将 $a$ 映射到代数 $(T a, \mu_a)$,而 $\varepsilon$ 选择求值器,我们最终得到 $\mu_a$。在左侧与 $U^T$ 的水平组合不会改变任何内容,因为 $U^T$ 对态射的作用是平凡的。所以,确实由此伴随关系得到的 $\mu$ 与原始 Monad $T$ 的 $\mu$ 是相同的。

\section{Kleisli 范畴(The Kleisli Category)}

我们之前已经见过 Kleisli 范畴。这是从另一个范畴 $\cat{C}$ 和一个 Monad $T$ 构建的范畴。我们将此范畴称为 $\cat{C}_T$。Kleisli 范畴 $\cat{C}_T$ 中的对象是 $\cat{C}$ 的对象,但态射有所不同。Kleisli 范畴中从 $a$ 到 $b$ 的态射 $f_{\cat{K}}$ 对应于原始范畴中从 $a$ 到 $T b$ 的态射 $f$。我们称这个态射为从 $a$ 到 $b$ 的 Kleisli 箭头。

Kleisli 范畴中态射的组合是通过 Kleisli 箭头的 Monad 组合定义的。例如,让我们在 Kleisli 范畴中组合 $g_{\cat{K}}$ 和 $f_{\cat{K}}$。在 Kleisli 范畴中我们有:
\begin{gather*}
  f_{\cat{K}} \Colon a \to b \\
  g_{\cat{K}} \Colon b \to c
\end{gather*}
这在范畴 $\cat{C}$ 中对应于:
\begin{gather*}
  f \Colon a \to T b \\
  g \Colon b \to T c
\end{gather*}
我们将组合定义为:
\[h_{\cat{K}} = g_{\cat{K}} \circ f_{\cat{K}}\]
作为 $\cat{C}$ 中的一个 Kleisli 箭头
\begin{align*}
  h & \Colon a \to T c          \\
  h & = \mu \circ (T g) \circ f
\end{align*}
在 Haskell 中我们可以这样写:

\src{snippet05}
从 $\cat{C}$ 到 $\cat{C}_T$ 存在一个函子 $F$,它在对象上是平凡的。在态射上,它将 $\cat{C}$ 中的态射 $f$ 映射到 $\cat{C}_T$ 中的态射,方法是创建一个将 $f$ 的返回值装饰(embellish)的 Kleisli 箭头。给定一个态射:
\[f \Colon a \to b\]
它在 $\cat{C}_T$ 中创建一个具有相应 Kleisli 箭头的态射:
\[\eta \circ f\]
在 Haskell 中我们可以这样写:

\src{snippet06}
我们还可以定义一个从 $\cat{C}_T$ 回到 $\cat{C}$ 的函子 $G$。它将 Kleisli 范畴中的对象 $a$ 映射到 $\cat{C}$ 中的对象 $T a$。它对 Kleisli 箭头 $f_{\cat{K}}$(对应的态射为:
\[f \Colon a \to T b\]
)在 $\cat{C}$ 中的态射是:
\[T a \to T b\]
给定的定义是先提升 $f$,然后应用 $\mu$:
\[\mu_{T b} \circ T f\]
在 Haskell 表示中,这样写:

\begin{snipv}
  G f\textsubscript{T} = join . fmap f
\end{snipv}
你可能认出这是用 \code{join} 定义 Monad bind 的方式。

很容易看出这两个函子形成了一个伴随关系:
\[F \dashv G\]
并且它们的组合 $G \circ F$ 再现了原始 Monad $T$。

因此,这是第二个生成相同 Monad 的伴随关系。实际上,有一个范畴 $\cat{Adj}(\cat{C}, T)$,其中的伴随关系都产生相同的 Monad $T$ 在范畴 $\cat{C}$ 上。我们刚刚看到的 Kleisli 伴随是这个范畴中的初始对象,而 Eilenberg-Moore 伴随是终端对象。

\section{Comonad 的余代数(Coalgebras for Comonads)}

类似的构造可以应用于任何 \hyperref[comonads]{comonad} $W$。我们可以定义一个与 comonad 兼容的余代数(coalgebra)范畴。它们使以下交换图成立:

\begin{figure}[H]
  \centering
  \begin{subfigure}
    \centering
    \begin{tikzcd}[column sep=large, row sep=large]
      a \arrow[rd, equal]
      & Wa \arrow[l, "\epsilon_a"'] \\
      & a \arrow[u, "\mathit{coa}"']
    \end{tikzcd}
  \end{subfigure}%
  \hspace{1cm}
  \begin{subfigure}
    \centering
    \begin{tikzcd}[column sep=large, row sep=large]
      W(Wa)
      & Wa \arrow[l, "W\ \mathit{coa}"'] \\
      Wa \arrow[u, "\delta_a"]
      & a \arrow[u, "\mathit{coa}"] \arrow[l, "\mathit{coa}"']
    \end{tikzcd}
  \end{subfigure}
\end{figure}

\noindent
其中,$\mathit{coa}$ 是载体为 $a$ 的余代数的余求值态射:
\[\mathit{coa} \Colon a \to W a\]
$\varepsilon$ 和 $\delta$ 是定义 comonad 的两个自然变换(在 Haskell 中,它们的分量分别称为 \code{extract} 和 \code{duplicate})。

从这些余代数范畴到 $\cat{C}$ 存在一个明显的遗忘函子 $U^W$。它只会遗忘余求值态射。我们将考虑它的右伴随 $F^W$。
\[U^W \dashv F^W\]
遗忘函子的右伴随称为余自由函子(cofree functor)。$F^W$ 生成余自由余代数。它将 $\cat{C}$ 中的对象 $a$ 分配到余代数 $(W a, \delta_a)$。该伴随关系通过组合 $U^W \circ F^W$ 再现了原始 comonad。

类似地,我们可以构建一个带有 co-Kleisli 箭头的 co-Kleisli 范畴,并通过相应的伴随关系重新生成 comonad。

\section{Lenses}

让我们回到对 lenses 的讨论。一个 lens 可以写作一个余代数:
\[\mathit{coalg}_s \Colon a \to \mathit{Store}\ s\ a\]
对于函子 $\mathit{Store}\ s$:

\src{snippet07}
这个余代数也可以表示为一对函数:
\begin{align*}
  \mathit{set} & \Colon a \to s \to a \\
  \mathit{get} & \Colon a \to s
\end{align*}
(可以将 $a$ 理解为 “全体”(all),将 $s$ 理解为其“小”部分。)在这对函数的术语中,我们有:
\[\mathit{coalg}_s\ a = \mathit{Store}\ (\mathit{set}\ a)\ (\mathit{get}\ a)\]
这里,$a$ 是类型 $a$ 的一个值。注意部分应用的 \code{set} 是一个从 $s$ 到 $a$ 的函数。

我们还知道 $\mathit{Store}\ s$ 是一个 comonad:

\src{snippet08}
问题是:在什么条件下,lens 是这个 comonad 的余代数?第一个一致性条件:
\[\varepsilon_a \circ \mathit{coalg} = \idarrow[a]\]
可以翻译为:
\[\mathit{set}\ a\ (\mathit{get}\ a) = a\]
这是 lens 定律,它表达了这样一个事实:如果你将结构 $a$ 的字段设置为其先前的值,那么什么也不会改变。

第二个条件:
\[\mathit{fmap}\ \mathit{coalg} \circ \mathit{coalg} = \delta_a \circ \mathit{coalg}\]
需要更多的工作。首先,回忆 \code{Store} 函子的 \code{fmap} 定义:

\src{snippet09}
将 \code{fmap coalg} 应用于 \code{coalg} 的结果会给出:

\src{snippet10}
另一方面,将 \code{duplicate} 应用于 \code{coalg} 的结果会生成:

\src{snippet11}
要使这两个表达式相等,\code{Store} 下的两个函数在作用于任意 \code{s} 时必须相等:

\src{snippet12}
展开 \code{coalg},我们得到:

\src{snippet13}
这相当于剩下的两个 lens 定律。第一个是:

\src{snippet14}
告诉我们,设置字段值两次与设置一次是相同的。第二个定律:

\src{snippet15}
告诉我们,获取字段的值将字段设置为 $s$ 会返回 $s$。

换句话说,一个良构的 lens 确实是 \code{Store} 函子的 comonad 余代数。

\section{Challenges}

\begin{enumerate}
  \tightlist
  \item
  讨论自由函子 $F \Colon C \to C^T$ 在态射上的作用。提示:使用 Monad $\mu$ 的自然性条件。
  \item
  定义伴随关系:
  \[U^W \dashv F^W\]
  \item
  证明上述伴随关系再现了原始 comonad。
\end{enumerate}
