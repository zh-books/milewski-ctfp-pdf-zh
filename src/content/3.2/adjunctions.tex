% !TEX root = ../../ctfp-print.tex

\lettrine[lhang=0.17]{在}{数学中,我们有}多种方式来表达一件事物与另一件事物相似。最严格的方式是相等(equality)。如果没有任何方式可以区分两个事物,那么它们就是相等的。在任何可以想象的情况下,一个可以被替换为另一个。例如,你是否注意到,每当我们谈到交换图(commuting diagrams)时,我们使用的是态射(morphisms)的\newterm{相等性}?这是因为态射形成一个集合(同态集(hom-set)),而集合的元素可以进行相等性比较。

但相等性往往过于严格。有许多事例表明,事物在所有实际用途上都是相同的,但实际上并不相等。例如,类型对 \code{(Bool, Char)} 严格来说并不等于 \code{(Char, Bool)},但我们理解它们包含相同的信息。这个概念最好用两个类型之间的\newterm{同构(isomorphism)}来描述——这是一个可逆的态射。既然它是一个态射,它就保留了结构;而“同构”意味着它是一个往返行程的一部分,无论从哪一边开始,都会回到原来的位置。在成对类型的情况下,这个同构称为 \code{swap}:

\src{snippet01}
\code{swap} 恰好是它自己的逆。

\section{伴随和单位/余单位对(Adjunction and Unit/Counit Pair)}

当我们谈论范畴是同构的时,我们用范畴之间的映射来表达,也就是函子(functors)。我们希望能够说两个范畴 $\cat{C}$ 和 $\cat{D}$ 是同构的,如果存在一个从 $\cat{C}$ 到 $\cat{D}$ 的函子 $R$(“右”),它是可逆的。换句话说,存在另一个从 $\cat{D}$ 回到 $\cat{C}$ 的函子 $L$(“左”),当它与 $R$ 组合时,等于恒等函子 $I$。有两种可能的组合,$R \circ L$ 和 $L \circ R$;以及两个可能的恒等函子:一个在 $\cat{C}$ 中,另一个在 $\cat{D}$ 中。

\begin{figure}[H]
  \centering
  \includegraphics[width=0.5\textwidth]{images/adj-1.jpg}
\end{figure}

\noindent
但这里有一个棘手的问题:两个函子\emph{相等}是什么意思?我们所说的相等是指:
\[R \circ L = I_{\cat{D}}\]
或者这个:
\[L \circ R = I_{\cat{C}}\]
将函子的相等性定义为对象的相等性是合理的。当两个函子作用于相等的对象时,它们应该产生相等的对象。但在一般情况下,我们没有在任意范畴中定义对象相等性的概念。它不属于范畴定义的一部分。(如果我们深入探讨“相等性到底是什么”这个问题,我们将进入同伦类型论(Homotopy Type Theory)这个深渊。)

你可能会认为函子是范畴范畴中的态射,因此它们应该是可以进行相等性比较的。确实,只要我们讨论的是小范畴,其中的对象形成一个集合,我们确实可以使用集合元素的相等性来比较对象的相等性。

但请记住,$\Cat$ 实际上是一个 $\cat{2}$-范畴。在一个 $\cat{2}$-范畴中,同态集(hom-sets)具有附加结构——在 $1$-态射之间存在 $2$-态射。在 $\Cat$ 中,$1$-态射是函子,而 $2$-态射是自然变换。因此,当谈到函子时,考虑自然同构作为相等的替代是更自然的(避免不了这个双关语!)。

所以,与其考虑范畴的同构,不如考虑更广泛的\newterm{等价(equivalence)}概念。如果我们能找到两个函子在它们之间来回映射,并且它们的组合(任意一种方式)\newterm{自然同构}于恒等函子,我们就说两个范畴 $\cat{C}$ 和 $\cat{D}$ 是\emph{等价的}。换句话说,在 $R \circ L$ 与恒等函子 $I_{\cat{D}}$ 之间存在一个双向自然变换,在 $L \circ R$ 与恒等函子 $I_{\cat{C}}$ 之间也存在一个双向自然变换。

伴随(Adjunction)甚至比等价更弱,因为它不要求两个函子的组合\emph{同构}于恒等函子。相反,它规定存在从 $I_{\cat{D}}$ 到 $R \circ L$ 的\newterm{单向}自然变换,以及从 $L \circ R$ 到 $I_{\cat{C}}$ 的另一个自然变换。以下是这两个自然变换的形式:
\begin{gather*}
  \eta \Colon I_{\cat{D}} \to R \circ L \\
  \varepsilon \Colon L \circ R \to I_{\cat{C}}
\end{gather*}
$\eta$ 被称为单位(unit),而 $\varepsilon$ 被称为余单位(counit)。

注意这两个定义之间的不对称性。一般来说,我们没有两个剩余的映射:
\begin{gather*}
  R \circ L \to I_{\cat{D}} \quad\quad\text{不一定} \\
  I_{\cat{C}} \to L \circ R \quad\quad\text{不一定}
\end{gather*}
由于这种不对称性,函子 $L$ 被称为 $R$ 的\newterm{左伴随},而函子 $R$ 是 $L$ 的右伴随(right adjoint)。当然,左和右只有当你以特定方式绘制图表时才有意义。

伴随的紧凑表示是:
\[L \dashv R\]
为了更好地理解伴随,让我们更详细地分析单位和余单位。

\begin{figure}[H]
  \centering
  \includegraphics[width=0.5\textwidth]{images/adj-unit.jpg}
\end{figure}

\noindent
让我们从单位开始。它是一个自然变换,因此它是一组态射的集合。给定一个在 $\cat{D}$ 中的对象 $d$,$\eta$ 的分量是 $I d$ 和 $(R \circ L) d$ 之间的态射;在图中,它被称为 $d'$:
\[\eta_d \Colon d \to (R \circ L) d\]
注意,组合 $R \circ L$ 是在 $\cat{D}$ 中的一个自函子(endofunctor)。

这个方程告诉我们,我们可以选择 $\cat{D}$ 中的任何对象 $d$ 作为起点,并使用往返函子 $R \circ L$ 选择目标对象 $d'$。然后我们发射一支箭——态射 $\eta_d$——到达目标。

\begin{figure}[H]
  \centering
  \includegraphics[width=0.5\textwidth]{images/adj-counit.jpg}
\end{figure}

\noindent
同样地,余单位 $\varepsilon$ 的分量可以描述为:
\[\varepsilon_{c} \Colon (L \circ R) c \to c\]
它告诉我们,我们可以选择 $\cat{C}$ 中的任何对象 $c$ 作为目标,并使用往返函子 $L \circ R$ 选择源对象 $c' = (L \circ R) c$。然后我们发射一支箭——态射 $\varepsilon_{c}$——从源对象到目标对象。

另一种看待单位和余单位的方式是,单位让我们在 $\cat{D}$ 上的任何地方\emph{引入}组合 $R \circ L$,而余单位让我们\emph{消除}组合 $L \circ R$,用 $\cat{C}$ 上的恒等函子替换它。这导致了一些“显而易见”的一致性条件,这些条件确保引入和消除的顺序不会改变任何东西:
\begin{gather*}
  L = L \circ I_{\cat{D}} \to L \circ R \circ L \to I_{\cat{C}} \circ L = L \\
  R = I_{\cat{D}} \circ R \to R \circ L \circ R \to R \circ I_{\cat{C}} = R
\end{gather*}
这些被称为三角恒等式(triangular identities),因为它们使以下图表交换:

\begin{figure}[H]
  \centering

  \begin{subfigure}
    \centering
    \begin{tikzcd}[column sep=large, row sep=large]
      L \arrow[rd, equal] \arrow[r, "L \circ \eta"]
      & L \circ R \circ L \arrow[d, "\epsilon \circ L"] \\
      & L
    \end{tikzcd}
  \end{subfigure}%
  \hspace{1cm}
  \begin{subfigure}
    \centering
    \begin{tikzcd}[column sep=large, row sep=large]
      R \arrow[rd, equal] \arrow[r, "\eta \circ R"]
      & R \circ L \circ R \arrow[d, "R \circ \epsilon"] \\
      & R
    \end{tikzcd}
  \end{subfigure}
\end{figure}

\noindent
这些是在函子范畴中的图:箭头是自然变换,它们的组合是自然变换的水平组合。用分量来表示,这些恒等式变为:
\begin{gather*}
  \varepsilon_{L d} \circ L \eta_d = \id_{L d} \\
  R \varepsilon_{c} \circ \eta_{R c} = \id_{R c}
\end{gather*}
我们经常在 Haskell 中以不同的名称看到单位和余单位。单位被称为 \code{return}(或在 \code{Applicative} 定义中称为 \code{pure}):

\src{snippet02}
而余单位则称为 \code{extract}:

\src{snippet03}
在这里,\code{m} 是对应于 $R \circ L$ 的(自)函子,而 \code{w} 是对应于 $L \circ R$ 的(自)函子。正如我们稍后会看到的,它们分别是单子(monad)和余单子(comonad)定义的一部分。

如果你将自函子视为一个容器,单位(或 \code{return})是一个多态函数,它为任意类型的值创建一个默认的盒子。余单位(或 \code{extract})则是相反的:它从容器中检索或生成单一值。

我们稍后会看到,每对伴随函子都会定义一个单子和一个余单子。反过来,每个单子或余单子都可以分解为一对伴随函子——尽管这种分解不是唯一的。

在 Haskell 中,我们经常使用单子,但很少将它们分解为一对伴随函子,主要是因为这些函子通常会将我们带出 $\Hask$。

然而,我们可以在 Haskell 中定义\newterm{自函子}的伴随关系。以下是取自 \code{Data.Functor.Adjunction} 的部分定义:

\src{snippet04}
此定义需要一些解释。首先,它描述了一个多参数类型类——这两个参数是 \code{f} 和 \code{u}。它建立了 \code{Adjunction} 之间的关系。

纵向条件(在竖线之后)指定了功能依赖性。例如,\code{f -> u} 意味着 \code{u} 是由 \code{f} 决定的(\code{f} 和 \code{u} 之间的关系是一个函数,这里是在类型构造函数上)。相反,\code{u -> f} 意味着如果我们知道 \code{u},那么 \code{f} 是唯一确定的。

稍后我将解释为什么在 Haskell 中,我们可以施加条件使得右伴随 \code{u} 是一个\newterm{可表函子}。

\section{伴随和同态集(Adjunctions and Hom-Sets)}

有一个等价的定义是基于同态集自然同构的伴随定义。这个定义与我们到目前为止研究的通用构造非常吻合。每次你听到某个唯一态射的声明,它因某种结构而因子化时,你应该将其视为从某个集合到同态集的映射。这就是“选择唯一态射”的意义。

此外,因子化通常可以通过自然变换来描述。因子化涉及交换图——某个态射等于两个态射(因子)的组合。自然变换将态射映射到交换图上。因此,在通用构造中,我们从态射到交换图,然后到唯一态射。我们最终得到的是从态射到态射的映射,或者从一个同态集到另一个(通常是在不同范畴中)。如果这个映射是可逆的,并且它可以自然地扩展到所有同态集上,我们就有了一个伴随。

通用构造和伴随之间的主要区别在于后者是全局定义的——适用于所有同态集。例如,使用通用构造你可以定义两个特定对象的积,即使在该范畴中的其他对象对之间不存在积。正如我们即将看到的,如果在一个范畴中\emph{任意一对}对象的积存在,它也可以通过伴随来定义。

\begin{figure}[H]
  \centering
  \includegraphics[width=0.5\textwidth]{images/adj-homsets.jpg}
\end{figure}

\noindent
这是使用同态集定义伴随的另一种方式。与之前一样,我们有两个函子 $L \Colon \cat{D} \to \cat{C}$ 和 $R \Colon \cat{C} \to \cat{D}$。我们选择两个任意的对象:在 $\cat{D}$ 中的源对象 $d$ 和在 $\cat{C}$ 中的目标对象 $c$。我们可以使用 $L$ 将源对象 $d$ 映射到 $\cat{C}$。现在我们在 $\cat{C}$ 中有两个对象,$L d$ 和 $c$。它们定义了一个同态集:
\[\cat{C}(L d, c)\]
类似地,我们可以使用 $R$ 将目标对象 $c$ 映射到 $\cat{D}$。现在我们在 $\cat{D}$ 中有两个对象,$d$ 和 $R c$。它们也定义了一个同态集:
\[\cat{D}(d, R c)\]
我们说 $L$ 是 $R$ 的左伴随当且仅当存在一个同态集之间的同构:
\[\cat{C}(L d, c) \cong \cat{D}(d, R c)\]
这个同构在 $d$ 和 $c$ 上都是自然的。自然性意味着源对象 $d$ 可以在 $\cat{D}$ 中平滑变化;目标对象 $c$ 可以在 $\cat{C}$ 中平滑变化。更精确地说,我们有一个自然变换 $\varphi$ 在以下两个(协变)函子之间,从 $\cat{C}$ 到 $\Set$。这是这些函子在对象上的作用:
\begin{gather*}
  c \to \cat{C}(L d, c) \\
  c \to \cat{D}(d, R c)
\end{gather*}
另一个自然变换 $\psi$ 作用在以下(逆变)函子之间:
\begin{gather*}
  d \to \cat{C}(L d, c) \\
  d \to \cat{D}(d, R c)
\end{gather*}
这两个自然变换都必须是可逆的。

很容易证明这两个伴随定义是等价的。例如,让我们从同态集同构开始推导单位变换:
\[\cat{C}(L d, c) \cong \cat{D}(d, R c)\]
由于这个同构适用于任何对象 $c$,它也必须适用于 $c = L d$:
\[\cat{C}(L d, L d) \cong \cat{D}(d, (R \circ L) d)\]
我们知道,左边至少必须包含一个态射,恒等态射。自然变换将这个态射映射到 $\cat{D}(d, (R \circ L) d)$ 中的一个元素,或者插入恒等函子 $I$,即:
\[\cat{D}(I d, (R \circ L) d)\]
我们得到一个由 $d$ 参数化的态射族。它们构成了从函子 $I$ 到函子 $R \circ L$ 之间的自然变换(自然性条件很容易验证)。这正是我们的单位 $\eta$。

反过来,从单位和余单位的存在开始,我们可以定义同态集之间的变换。例如,我们选择一个在同态集 $\cat{C}(L d, c)$ 中的任意态射 $f$。我们想定义一个 $\varphi$,它作用在 $f$ 上,生成一个在 $\cat{D}(d, R c)$ 中的态射。

实际上我们没有太多选择。我们可以尝试使用 $R$ 提升 $f$。这将生成一个态射 $R f$,从 $R (L d)$ 到 $R c$——这是 $\cat{D}((R \circ L) d, R c)$ 中的一个态射。

我们需要的是一个 $\varphi$ 的分量,一个从 $d$ 到 $R c$ 的态射。这不是问题,因为我们可以使用 $\eta_d$ 的一个分量从 $d$ 到 $(R \circ L) d$。我们得到:
\[\varphi_f = R f \circ \eta_d\]
另一个方向类似,$\psi$ 的推导也是如此。

回到 Haskell 中 \code{Adjunction} 的定义,自然变换 $\varphi$ 和 $\psi$ 被分别替换为多态的(在 \code{a} 和 \code{b} 上)函数 \code{leftAdjunct} 和 \code{rightAdjunct}。函子 $L$ 和 $R$ 分别被称为 \code{f} 和 \code{u}:

\src{snippet05}
在 \code{unit}/\code{counit} 表述和 \code{leftAdjunct}/\allowbreak\code{rightAdjunct} 表述之间的等价性由以下映射见证:

\src{snippet06}
从伴随的范畴描述到 Haskell 代码的转换是非常有启发性的。我强烈建议将其作为练习。

我们现在准备解释为什么在 Haskell 中右伴随自动是一个\hyperref[representable-functors]{可表函子}。原因是,在第一近似下,我们可以将 Haskell 类型的范畴视为集合的范畴。

当右范畴 $\cat{D}$ 是 $\Set$ 时,右伴随 $R$ 是一个从 $\cat{C}$ 到 $\Set$ 的函子。如果我们可以找到 $\cat{C}$ 中的一个对象 $\mathit{rep}$,使得同态函子 $\cat{C}(\mathit{rep}, \_)$ 自然同构于 $R$,则该函子是可表的。事实证明,如果 $R$ 是某个函子 $L$ 从 $\Set$ 到 $\cat{C}$ 的右伴随,这样的对象总是存在——它是单集合 $()$ 在 $L$ 下的像:
\[\mathit{rep} = L ()\]
确实,伴随告诉我们,以下两个同态集是自然同构的:
\[\cat{C}(L (), c) \cong \Set((), R c)\]
对于给定的 $c$,右侧是从单集合 $()$ 到 $R c$ 的函数集。我们前面已经看到,每个这样的函数从 $R c$ 中选择一个元素。这些函数的集合与 $R c$ 集合同构。因此我们有:
\[\cat{C}(L (), -) \cong R\]
这表明 $R$ 确实是可表的。

\section{从伴随到积(Product from Adjunction)}

我们之前已经使用通用构造介绍了几个概念。当这些概念在全局定义时,使用伴随表达起来更容易。最简单的非平凡例子是积(product)。\hyperref[products-and-coproducts]{积的通用构造}的要点是能够通过通用积因子化任何积状候选对象。

更确切地说,两个对象 $a$ 和 $b$ 的积是对象 $(a\times{}b)$(或在 Haskell 中表示为 \code{(a, b)}),配有两个态射 $\mathit{fst}$ 和 $\mathit{snd}$,使得对于任何其他候选对象 $c$,配有两个态射 $p \Colon c \to a$ 和 $q \Colon c \to b$,存在一个唯一的态射 $m \Colon c \to (a, b)$,它因子化了 $p$ 和 $q$ 通过 $\mathit{fst}$ 和 $\mathit{snd}$。

如我们前面所见,在 Haskell 中,我们可以实现一个 \code{factorizer},它从两个投影生成这个态射:

\src{snippet07}
验证因子化条件成立是很容易的:

\src{snippet08}
我们有一个映射,它接受一对态射 \code{p} 和 \code{q} 并生成另一个态射 \code{m = factorizer p q}。

我们如何将其转化为定义伴随所需的同态集之间的映射?诀窍是走出 $\Hask$,并将态射对视为在积范畴中的单个态射。

让我提醒你,什么是积范畴。取两个任意范畴 $\cat{C}$ 和 $\cat{D}$。积范畴 $\cat{C}\times{}\cat{D}$ 中的对象是对象对,一个来自 $\cat{C}$,另一个来自 $\cat{D}$。态射是态射对,一个来自 $\cat{C}$,一个来自 $\cat{D}$。

为了在某个范畴 $\cat{C}$ 中定义积,我们应该从积范畴 $\cat{C}\times{}\cat{C}$ 开始。来自 $\cat{C}$ 的态射对是在积范畴 $\cat{C}\times{}\cat{C}$ 中的单个态射。

\begin{figure}[H]
  \centering
  \includegraphics[width=0.5\textwidth]{images/adj-productcat.jpg}
\end{figure}

\noindent
起初使用积范畴来定义积可能有点令人困惑。然而,这些积是非常不同的。我们不需要通用构造来定义积范畴。我们只需要对象对和态射对的概念。

然而,来自 $\cat{C}$ 的对象对\emph{不是} $\cat{C}$ 中的对象。它是一个不同范畴 $\cat{C}\times{}\cat{C}$ 中的对象。我们可以将该对形式化为 $\langle a, b \rangle$,其中 $a$ 和 $b$ 是 $\cat{C}$ 的对象。另一方面,通用构造是必要的,以便在\emph{相同}的范畴 $\cat{C}$ 中定义对象 $a\times{}b$(或在 Haskell 中表示为 \code{(a, b)})。该对象应根据通用构造表示对 $\langle a, b \rangle$。它不总是存在,并且即使它存在于某些情况下,也可能不适用于 $\cat{C}$ 中的其他对象对。

现在让我们将 \code{factorizer} 看作同态集的映射。第一个同态集在积范畴 $\cat{C}\times{}\cat{C}$ 中,第二个在 $\cat{C}$ 中。积范畴中的一般态射是态射对 $\langle f, g \rangle$:
\begin{gather*}
  f \Colon c' \to a \\
  g \Colon c'' \to b
\end{gather*}
其中 $c''$ 可能不同于 $c'$。但为了定义积,我们对 $\cat{C}\times{}\cat{C}$ 中的一个特殊态射感兴趣,这对 $p$ 和 $q$ 共享相同的源对象 $c$。这没问题:在伴随的定义中,左同态集的源不是任意对象——它是左函子 $L$ 在右范畴的某个对象上作用的结果。符合条件的函子很容易猜测——它是从 $\cat{C}$ 到 $\cat{C}\times{}\cat{C}$ 的对角函子(diagonal functor)$\Delta$,其对对象的作用是:
\[\Delta c = \langle c, c \rangle\]
我们伴随中的左侧同态集应为:
\[(\cat{C}\times{}\cat{C})(\Delta c, \langle a, b \rangle)\]
它是积范畴中的同态集。其元素是我们识别为 \code{factorizer} 参数的态射对:
\[(c \to a) \to (c \to b) \ldots{}\]
右侧同态集存在于 $\cat{C}$ 中,它在源对象 $c$ 和某些函子 $R$ 的结果之间映射。该函子将对 $\langle a, b \rangle$ 映射到我们的积对象 $a\times{}b$。我们将该同态集的元素识别为 \code{factorizer} 的结果:
\[\ldots{} \to (c \to (a, b))\]

\begin{figure}[H]
  \centering
  \includegraphics[width=0.5\textwidth]{images/adj-product.jpg}
\end{figure}

\noindent
我们仍然没有完整的伴随。为此,我们首先需要 \code{factorizer} 是可逆的——我们正在构建同态集之间的同构。逆的 \code{factorizer} 应该从某个对象 $c$ 到积对象 $a\times{}b$ 的态射 $m$ 开始。换句话说,$m$ 应该是:
\[\cat{C}(c, a\times{}b)\]
逆因子应该将 $m$ 映射到 $\cat{C}\times{}\cat{C}$ 中的态射对 $\langle p, q \rangle$,该对从 $\langle c, c \rangle$ 到 $\langle a, b \rangle$;换句话说,该态射是:
\[(\cat{C}\times{}\cat{C})(\Delta\ c, \langle a, b \rangle)\]
如果该映射存在,则我们得出结论对角函子的右伴随存在。该函子定义一个积。

在 Haskell 中,我们可以通过分别与 \code{fst} 和 \code{snd} 组合构造 \code{factorizer} 的逆函数。

\begin{snip}{haskell}
  p = fst . m
  q = snd . m
\end{snip}
要完成两种定义积方式的等价性证明,我们还需要证明同态集之间的映射在 $a$,$b$ 和 $c$ 上是自然的。我将此留作给有志读者的练习。

总结我们所做的:一个范畴积可以全局定义为对角函子的\newterm{右伴随}:
\[(\cat{C}\times{}\cat{C})(\Delta c, \langle a, b \rangle) \cong \cat{C}(c, a\times{}b)\]
这里 $a\times{}b$ 是我们右伴随函子 $\mathit{Product}$ 对对象对 $\langle a, b \rangle$ 的作用结果。请注意,任何从 $\cat{C}\times{}\cat{C}$ 到 $\cat{C}$ 的函子都是一个双函子(bifunctor),因此 $\mathit{Product}$ 是一个双函子。在 Haskell 中,$\mathit{Product}$ 双函子简单地表示为 \code{(,)}。你可以将它应用于两个类型并得到它们的积类型,例如:

\src{snippet09}

\section{从伴随到指数对象(Exponential from Adjunction)}

指数对象 $b^a$ 或函数对象 $a \Rightarrow b$ 可以使用\hyperref[function-types]{通用构造}来定义。如果这种构造存在于所有对象对中,它可以看作一个伴随。再一次,诀窍是集中注意以下声明:

\begin{quote}
  对于任何其他对象 $z$ 及其态射 $g \Colon z\times{}a \to b$,存在一个唯一的态射 $h \Colon z \to (a \Rightarrow b)$
\end{quote}
该声明建立了同态集之间的映射。

在这种情况下,我们处理的是同一范畴中的对象,因此两个伴随函子是自函子。左(自)函子 $L$ 作用在对象 $z$ 上生成 $z\times{}a$。这是一个对应于固定某个 $a$ 的积函子。

右(自)函子 $R$ 作用在对象 $b$ 上生成函数对象 $a \Rightarrow b$(或 $b^a$)。再一次,$a$ 是固定的。该伴随函子通常写为:
\[-\times{}a \dashv (-)^a\]
该伴随函子下同态集的映射在我们用通用构造的图中最好体现出来。

\begin{figure}[H]
  \centering
  \includegraphics[width=0.4\textwidth]{images/adj-expo.jpg}
\end{figure}

\noindent
注意到 $\mathit{eval}$ 态射\footnote{参见第 9 章 \hyperref[function-types]{通用构造}。}其实就是这个伴随的余单位:
\[(a \Rightarrow b)\times{}a \to b\]
其中:
\[(a \Rightarrow b)\times{}a = (L \circ R) b\]
我之前提到过,通用构造定义了一个唯一的对象,至多是同构的。这就是为什么我们有“积”和“指数对象”。这个属性也适用于伴随:如果一个函子有一个伴随,这个伴随至多是同构的。

\section{挑战}

\begin{enumerate}
  \tightlist
  \item
  推导出自然变换 $\psi$ 的自然性方框,该变换作用于以下两个(逆变)函子之间:
  \begin{gather*}
    a \to \cat{C}(L a, b) \\
    a \to \cat{D}(a, R b)
  \end{gather*}
  \item
  从伴随的第二个定义中同态集同构开始推导余单位 $\varepsilon$。
  \item
  完成伴随的两种定义等价性的证明。
  \item
  证明可以通过一个伴随来定义余积。从余积的因子化定义开始。
  \item
  证明余积是对角函子的左伴随。
  \item
  在 Haskell 中定义一个积与函数对象之间的伴随关系。
\end{enumerate}
