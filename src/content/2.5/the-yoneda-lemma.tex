% !TEX root = ../../ctfp-print.tex

\lettrine[lhang=0.17]{大}{多数范畴论中的构造}都是从数学中其他更具体的领域中推广而来的结果。像积(product)、余积(coproduct)、幺半群(monoids)、指数(exponentials)等概念早在范畴论之前就已经为人所知了。它们在不同的数学分支中可能有不同的名称。例如,在集合论中是笛卡尔积(Cartesian product)、在序理论中是交(meet)、在逻辑中是合取(conjunction)——这些都是范畴积这一抽象概念的具体例子。

在这方面,$\textbf{Yoneda 引理}$格外引人注目,因为它是关于范畴的一个广泛声明,在数学的其他分支中几乎没有前例。有人说它最接近的类比是群论中的凯莱定理(Cayley’s theorem),即每个群都同构于某个集合的置换群。

Yoneda 引理的背景是一个任意的范畴 $\cat{C}$,以及从 $\cat{C}$ 到 $\Set$ 的一个函子 $F$。我们在前一节中看到,一些 $\Set$ 值函子是可表示的,也就是说,它们同构于某个同态函子(hom-functor)。Yoneda 引理告诉我们,所有的 $\Set$ 值函子都可以通过自然变换(natural transformations)从同态函子中获得,并且它明确列举了所有这样的变换。

当我谈到自然变换时,我提到自然性条件可能非常具有约束性。当你在一个对象上定义了自然变换的一个分量时,自然性可能足够强大,可以将这个分量“传递”到通过态射与之连接的另一个对象中。在源范畴和目标范畴中的对象之间的箭头越多,你就有越多的约束条件来传递自然变换的分量。$\Set$ 恰好是一个箭头非常丰富的范畴。

Yoneda 引理告诉我们,两个同态函子和任意其他函子 $F$ 之间的自然变换完全由在一个点上的单个分量的值决定!其余的自然变换仅由自然性条件决定。

因此,让我们回顾一下 Yoneda 引理中涉及的两个函子之间的自然性条件。第一个函子是同态函子。它将 $\cat{C}$ 中的任意对象 $x$ 映射为 $\cat{C}(a, x)$ 的态射集,其中 $a$ 是 $\cat{C}$ 中的一个固定对象。我们还看到它将从 $x \to y$ 的任何态射 $f$ 映射为 $\cat{C}(a, f)$。

第二个函子是任意的 $\Set$ 值函子 $F$。

我们将这两个函子之间的自然变换称为 $\alpha$。由于我们在 $\Set$ 中操作,自然变换的分量,如 $\alpha_x$ 或 $\alpha_y$,只是集合之间的常规函数:
\begin{gather*}
  \alpha_x \Colon \cat{C}(a, x) \to F x \\
  \alpha_y \Colon \cat{C}(a, y) \to F y
\end{gather*}

\begin{figure}[H]
  \centering
  \includegraphics[width=0.4\textwidth]{images/yoneda1.png}
\end{figure}

\noindent
因为这些只是函数,我们可以观察它们在特定点上的值。但是在集合 $\cat{C}(a, x)$ 中的一个点是什么?这是关键的观察点:集合 $\cat{C}(a, x)$ 中的每个点也是从 $a$ 到 $x$ 的态射 $h$。

因此,自然性方块 $\alpha$:
\[\alpha_y \circ \cat{C}(a, f) = F f \circ \alpha_x\]
在作用于 $h$ 时,逐点变为:
\[\alpha_y (\cat{C}(a, f) h) = (F f) (\alpha_x h)\]
你可能还记得在前一节中,同态函子 $\cat{C}(a,-)$ 作用于态射 $f$ 时的操作被定义为前组合:
\[\cat{C}(a, f) h = f \circ h\]
这就导致了:
\[\alpha_y (f \circ h) = (F f) (\alpha_x h)\]
这种条件的强大之处在于将其特化为 $x = a$ 的情况。

\begin{figure}[H]
  \centering
  \includegraphics[width=0.4\textwidth]{images/yoneda2.png}
\end{figure}

\noindent
在这种情况下,$h$ 变成了从 $a$ 到 $a$ 的态射。我们知道至少有一个这样的态射,即 $h = \id_a$。让我们将其代入:
\[\alpha_y f = (F f) (\alpha_a \id_a)\]
注意到刚刚发生的事情:左侧是自然变换 $\alpha_y$ 对 $\cat{C}(a, y)$ 中任意元素 $f$ 的作用。它完全由 $\alpha_a$ 在 $\id_a$ 上的单一值决定。我们可以选择任意这样的值,并且它将生成一个自然变换。由于 $\alpha_a$ 的值位于集合 $F a$ 中,因此 $F a$ 中的任何点都将定义一些 $\alpha$。

反之,给定从 $\cat{C}(a, -)$ 到 $F$ 的任意自然变换 $\alpha$,你可以在 $\id_a$ 处评估它以获得 $F a$ 中的一个点。

我们刚刚证明了 Yoneda 引理:

\begin{quote}
  从 $\cat{C}(a, -)$ 到 $F$ 的自然变换与 $F a$ 的元素之间存在一一对应关系。
\end{quote}
换句话说,
\[\cat{Nat}(\cat{C}(a, -), F) \cong F a\]
或者,如果我们使用符号 $[\cat{C}, \Set]$ 表示 $\cat{C}$ 和 $\Set$ 之间的函子范畴,则自然变换的集合只是该范畴中的一个同态集(hom-set),我们可以写作:
\[[\cat{C}, \Set](\cat{C}(a, -), F) \cong F a\]
稍后我会解释这个对应关系实际上是一个自然同构。

现在让我们试着对这个结果获得一些直觉。最令人惊奇的是,整个自然变换从一个成核点开始结晶:我们在 $\id_a$ 处赋予它的值。它从那个点开始扩展,遵循自然性条件。它渗透到 $\Set$ 中 $\cat{C}$ 的像中。因此,让我们首先考虑 $\cat{C}$ 在 $\cat{C}(a, -)$ 下的像是什么。

让我们从 $a$ 本身的像开始。在同态函子 $\cat{C}(a, -)$ 下,$a$ 被映射为 $\cat{C}(a, a)$ 的集合。在函子 $F$ 下,它被映射为集合 $F a$。自然变换 $\alpha_a$ 的分量是从 $\cat{C}(a, a)$ 到 $F a$ 的某个函数。让我们关注 $\cat{C}(a, a)$ 集合中的一个点,该点对应于态射 $\id_a$。为了强调它只是集合中的一个点,我们称它为 $p$。$\alpha_a$ 的分量应该将 $p$ 映射为 $F a$ 中的某个点 $q$。我会向你展示,任何 $q$ 的选择都会导致一个唯一的自然变换。

\begin{figure}[H]
  \centering
  \includegraphics[width=0.3\textwidth]{images/yoneda3.png}
\end{figure}

\noindent
第一个主张是,一个点 $q$ 的选择唯一地决定了其余的函数 $\alpha_a$。确实,让我们选择 $\cat{C}(a, a)$ 中的另一个点 $p'$,该点对应于从 $a$ 到 $a$ 的某个态射 $g$。这是 Yoneda 引理的魔力发生的地方:$g$ 可以看作是 $\cat{C}(a, a)$ 集合中的一个点 $p'$。同时,它选择了集合之间的两个函数。实际上,在同态函子下,态射 $g$ 被映射为函数 $\cat{C}(a, g)$;在 $F$ 下,它被映射为 $F g$。

\begin{figure}[H]
  \centering
  \includegraphics[width=0.4\textwidth]{images/yoneda4.png}
\end{figure}

\noindent
现在让我们考虑 $\cat{C}(a, g)$ 作用于我们原来的 $p$ 的动作,正如你所记得的那样,它对应于 $\id_a$。它被定义为前组合 $g \circ \id_a$,等于 $g$,即对应于我们的点 $p'$。因此,态射 $g$ 被映射为一个函数,当作用于 $p$ 时产生 $p'$,即 $g$。我们已经完成了完整的循环!

现在考虑 $F g$ 对 $q$ 的作用。它是 $F a$ 中的一个点 $q'$。为了完成自然性方块,$p'$ 必须在 $\alpha_a$ 下被映射为 $q'$。我们选择了任意的 $p'$(任意的 $g$),并推导出它在 $\alpha_a$ 下的映射。函数 $\alpha_a$ 因此完全确定。

第二个主张是,对于 $\cat{C}$ 中与 $a$ 相连的任何对象 $x$,$\alpha_x$ 是唯一确定的。推理是类似的,只不过现在我们有了两个更多的集合 $\cat{C}(a, x)$ 和 $F x$,从 $a$ 到 $x$ 的态射 $g$ 在同态函子下被映射为:
\[\cat{C}(a, g) \Colon \cat{C}(a, a) \to \cat{C}(a, x)\]
在 $F$ 下则被映射为:
\[F g \Colon F a \to F x\]
同样地,$\cat{C}(a, g)$ 作用于我们的 $p$ 是通过前组合给出的:$g \circ \id_a$,对应于 $\cat{C}(a, x)$ 中的一个点 $p'$。自然性决定了作用于 $p'$ 的 $\alpha_x$ 的值是:
\[q' = (F g) q\]
由于 $p'$ 是任意的,因此整个函数 $\alpha_x$ 因此确定。

\begin{figure}[H]
  \centering
  \includegraphics[width=0.4\textwidth]{images/yoneda5.png}
\end{figure}

\noindent
如果 $\cat{C}$ 中有与 $a$ 无连接的对象怎么办?它们都在 $\cat{C}(a, -)$ 下映射为单个集合——空集。回想一下,空集是集合范畴中的始对象(initial object)。这意味着从该集合到任何其他集合都有唯一的函数。我们称这个函数为 \code{absurd}。因此,在这里,自然变换的分量没有选择,它只能是 \code{absurd}。

理解 Yoneda 引理的一种方法是认识到,$\Set$ 值函子之间的自然变换只是函数的家族,而函数通常是有损的(lossy)。函数可能会丢失信息,并且可能只覆盖其值域的一部分。唯一不丢失信息的函数是可逆的——即同构(isomorphisms)。因此,最好的结构保留 $\Set$ 值函子是可表示的函子。它们要么是同态函子,要么是自然同构于同态函子的函子。任何其他函子 $F$ 都是通过有损变换从同态函子中获得的。这样的变换不仅可能丢失信息,而且可能只覆盖函子 $F$ 在 $\Set$ 中的像的一小部分。

\section{Haskell 中的 Yoneda 引理}

我们已经在 Haskell 中以 reader 函子的形式遇到过同态函子:

\src{snippet01}
reader 函子通过前组合映射态射(这里的函数):

\src{snippet02}
Yoneda 引理告诉我们,reader 函子可以自然地映射到任何其他函子。

自然变换是多态函数。因此,给定一个函子 \code{F},我们可以从 reader 函子映射到它:

\src{snippet03}
和往常一样,\code{forall} 是可选的,但我喜欢明确地写出来,以强调自然变换的参数多态性。

Yoneda 引理告诉我们,这些自然变换与 \code{F a} 的元素是一一对应的:

\begin{snipv}
  forall x . (a -> x) -> F x \ensuremath{\cong} F a
\end{snipv}
这个等式的右侧是我们通常认为的数据结构。记住函子作为广义容器的解释?\code{F a} 是一个 \code{a} 的容器。但左侧是一个多态函数,它接受一个函数作为参数。Yoneda 引理告诉我们,这两种表示是等价的——它们包含相同的信息。

换句话说:给我一个类型为:

\src{snippet04}
的多态函数,我将生成一个 \code{a} 的容器。诀窍是我们在证明 Yoneda 引理时使用的:我们用 \code{id} 调用这个函数来获得 \code{F a} 的一个元素:

\src{snippet05}
反之亦然:给定一个类型为 \code{F a} 的值:

\src{snippet06}
可以定义一个类型正确的多态函数:

\src{snippet07}
你可以轻松地在两种表示之间来回转换。

拥有多种表示的好处是,其中一种可能比另一种更容易组合,或者在某些应用中更高效。

这一原理的最简单示例是编译器构建中常用的代码转换:续延传递风格(continuation-passing style, \acronym{CPS})。这是 Yoneda 引理对恒等函子的最简单应用。将 \code{F} 替换为恒等函子得到:

\begin{snipv}
  forall r . (a -> r) -> r \ensuremath{\cong} a
\end{snipv}
这一公式的解释是,任何类型 \code{a} 都可以被一个接受 \code{a} 的“处理器”的函数替换。处理器是接受 \code{a} 并执行其余计算的函数——续延。(类型 \code{r} 通常封装某种状态代码。)

这种编程风格在用户界面(UIs)、异步系统和并发编程中非常常见。\acronym{CPS} 的缺点是它涉及控制的反转。代码分为生产者和消费者(处理器),并且不易组合。任何做过不小的 Web 编程的人都熟悉从交互式有状态处理器中产生的面条代码噩梦。正如我们稍后将看到的那样,合理使用函子和单子可以恢复 \acronym{CPS} 的某些组合属性。

\section{对偶 Yoneda 引理(Co-Yoneda)}

像往常一样,我们通过逆转箭头的方向获得了一个额外的构造。Yoneda 引理可以应用于对偶范畴 $\cat{C}^\mathit{op}$,为我们提供了逆变函子之间的映射。

同样地,我们可以通过固定同态函子的目标对象而不是源对象来推导对偶 Yoneda 引理。我们得到了从 $\cat{C}$ 到 $\Set$ 的逆变同态函子:
$\cat{C}(-, a)$。Yoneda 引理的逆变版本建立了从这个函子到任何其他逆变函子 $F$ 的自然变换与集合 $F a$ 的元素之间的一一对应关系:
\[\cat{Nat}(\cat{C}(-, a), F) \cong F a\]
这是 Haskell 版本的对偶 Yoneda 引理:

\begin{snipv}
  forall x . (x -> a) -> F x \ensuremath{\cong} F a
\end{snipv}
请注意,在某些文献中,这个逆变版本被称为 Yoneda 引理。

\section{挑战(Challenges)}

\begin{enumerate}
  \tightlist
  \item
  证明构成 Haskell 中 Yoneda 同构的两个函数 \code{phi} 和 \code{psi} 是彼此的逆。

  \begin{snip}{haskell}
    phi :: (forall x . (a -> x) -> F x) -> F a
    phi alpha = alpha id

    psi :: F a -> (forall x . (a -> x) -> F x)
    psi fa h = fmap h fa
  \end{snip}
  \item
  离散范畴是指只有对象而没有态射的范畴,除了恒等态射之外。对于从这样一个范畴来的函子,Yoneda 引理如何运作?
  \item
  单位类型的列表 \code{{[}(){]}} 不包含除长度之外的其他信息。因此,作为一种数据类型,它可以被视为整数的编码。一个空列表编码为零,一个单元素列表 \code{{[}(){]}}(值,而不是类型)编码为一,依此类推。使用 Yoneda 引理为列表函子构造这种数据类型的另一种表示。
\end{enumerate}

\section{参考文献(Bibliography)}

\begin{enumerate}
  \tightlist
  \item
  \urlref{https://www.youtube.com/watch?v=TLMxHB19khE}{Catsters} 视频。
\end{enumerate}
