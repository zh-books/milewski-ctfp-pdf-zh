% !TEX root = ../../ctfp-print.tex

\lettrine[lhang=0.17]{现}{在我们知道} monad 的作用是什么了——它让我们可以组合装饰过的函数——真正有趣的问题是,为什么装饰过的函数在函数式编程中如此重要。我们已经看到了一个例子,即 \code{Writer} monad,其中装饰使我们能够在多个函数调用中创建和累积日志。本来需要使用不纯函数(例如,通过访问和修改某个全局状态)来解决的问题,现在通过纯函数解决了。

\section{问题(The Problem)}

这里是一个类似问题的简短列表,摘自 \urlref{https://core.ac.uk/download/pdf/21173011.pdf}{Eugenio Moggi 的开创性论文},所有这些问题传统上都是通过放弃函数的纯性来解决的。

\begin{itemize}
  \tightlist
  \item
  不完全性:可能不会终止的计算
  \item
  非确定性:可能返回多个结果的计算
  \item
  副作用:访问/修改状态的计算

  \begin{itemize}
    \tightlist
    \item
    只读状态,或称为环境
    \item
    只写状态,或称为日志
    \item
    读/写状态
  \end{itemize}
  \item
  异常:可能失败的部分函数
  \item
  Continuations:保存程序状态并在需要时恢复的能力
  \item
  交互式输入
  \item
  交互式输出
\end{itemize}

真正令人惊讶的是,所有这些问题都可以用同一个巧妙的技巧解决:转向装饰过的函数。当然,每种情况下的装饰将完全不同。

你必须意识到,在这个阶段,并没有要求装饰是 monadic 的。只有当我们坚持组合——能够将一个装饰过的函数分解为更小的装饰过的函数时——我们才需要 monad。再一次,由于每种装饰都不同,monadic 组合将以不同的方式实现,但总体模式是相同的。这是一个非常简单的模式:具有结合性和单位元的组合。

下一节有大量 Haskell 示例。如果你迫切想回到范畴论,或者你已经熟悉 Haskell 的 monad 实现,可以随意略读甚至跳过它。

\section{解决方案(The Solution)}

首先,让我们分析一下我们如何使用 \code{Writer} monad。我们从一个执行特定任务的纯函数开始——给定参数,它会生成特定的输出。我们用另一个函数替换了这个函数,该函数通过将原始输出与字符串配对来装饰它。这就是我们解决日志记录问题的方法。

我们不能止步于此,因为通常我们不想处理单一的解决方案。我们需要能够将一个生成日志的函数分解为多个更小的生成日志的函数。正是这些较小函数的组合导致了 monad 概念的出现。

真正令人惊叹的是,装饰函数返回类型的相同模式适用于各种通常需要放弃纯性的复杂问题。让我们逐一浏览我们的列表,并确定适用于每个问题的装饰。

\subsection{不完全性(Partiality)}

我们通过将每个可能不会终止的函数的返回类型转化为“提升的”类型来修改它——一种包含原始类型的所有值加上特殊的“底”值 $\bot$ 的类型。例如,\code{Bool} 类型作为一个集合将包含两个元素:\code{True} 和 \code{False}。提升后的 \code{Bool} 包含三个元素。返回提升后的 \code{Bool} 的函数可能会生成 \code{True} 或 \code{False},也可能会永远执行下去。

有趣的是,在像 Haskell 这样的惰性语言中,一个永无止境的函数实际上可能会返回一个值,并且这个值可能会被传递给下一个函数。我们称这个特殊的值为底值。只要不显式需要这个值(例如,用于模式匹配或生成输出),它就可以被传递而不会阻碍程序的执行。因为每个 Haskell 函数可能都是非终止的,Haskell 中的所有类型都被假定为提升的。这就是为什么我们经常谈论 Haskell(提升的)类型和函数的范畴 $\Hask$,而不是更简单的 $\Set$。不过,是否可以明确 $\Hask$ 是一个真正的范畴仍不清楚(参见这篇 \urlref{http://math.andrej.com/2016/08/06/hask-is-not-a-category/}{Andrej Bauer 的文章})。

\subsection{非确定性(Nondeterminism)}

如果一个函数可以返回多个不同的结果,那么它也可以一次性返回所有结果。从语义上讲,一个非确定性函数等价于一个返回结果列表的函数。在像 Haskell 这样的惰性垃圾回收语言中,这很有意义。例如,如果你只需要一个值,你可以只取列表的头部,其余部分永远不会被计算。如果你需要一个随机值,使用随机数生成器选择列表中的第 n 个元素。惰性甚至允许你返回一个无限的结果列表。

在列表 monad——Haskell 的非确定性计算实现中——\code{join} 实现为 \code{concat}。请记住,\code{join} 应该将容器的容器压平——\code{concat} 将一个列表的列表连接成一个单一的列表。\code{return} 创建一个单元素列表:

\src{snippet01}
列表 monad 的 bind 运算符由通用公式给出:先 \code{fmap},然后 \code{join},在此情况下为:

\src{snippet02}
在这里,函数 \code{k} 本身生成一个列表,并应用于列表 \code{as} 的每个元素。结果是一个列表的列表,它通过 \code{concat} 进行压平。

从程序员的角度来看,处理一个列表比在循环中调用非确定性函数或实现一个返回迭代器的函数要容易得多(尽管在 \urlref{http://ericniebler.com/2014/04/27/range-comprehensions/}{现代 C++} 中,返回一个惰性范围几乎等同于在 Haskell 中返回一个列表)。

在游戏编程中创造性地使用非确定性是一个很好的例子。例如,当计算机与人类下棋时,它无法预测对手的下一步棋。但它可以生成所有可能的走法列表,并逐一分析它们。同样,非确定性解析器可以为给定表达式生成所有可能的解析列表。

尽管我们可以将返回列表的函数解释为非确定性的,但列表 monad 的应用范围要广得多。这是因为连接生成列表的计算是迭代结构(循环)的一个完美的函数式替代方案,循环通常在命令式编程中使用。单个循环通常可以使用 \code{fmap} 重新编写,以将循环体应用于列表的每个元素。列表 monad 中的 \code{do} 语法可以用来替换复杂的嵌套循环。

我最喜欢的例子是生成勾股数三元组的程序——可以构成直角三角形边的正整数三元组。

\src{snippet03}
第一行告诉我们,\code{z} 从正整数的无限列表 \code{{[}1..{]}} 中获取一个元素。然后,\code{x} 从列表 \code{{[}1..z{]}} 中获取一个元素,该列表包含从 1 到 \code{z} 的数字。最后,\code{y} 从 \code{x} 和 \code{z} 之间的数字列表中获取一个元素。我们可以使用三个数字 $1 \leqslant x \leqslant y \leqslant z$。函数 \code{guard} 接受一个 \code{Bool} 表达式,并返回一个单位列表:

\src{snippet04}
此函数(它是更大类 \code{MonadPlus} 的成员)用于过滤掉非勾股三元组。实际上,如果你查看 bind 运算符(或相关的 \code{>>} 运算符)的实现,你会注意到,当给定一个空列表时,它会生成一个空列表。另一方面,当给定一个非空列表(此处是包含单位 \code{{[}(){]}} 的单例列表)时,bind 将调用继续体,即 \code{return (x, y, z)},它生成一个验证的勾股三元组的单例列表。所有这些单例列表将通过包含的 binds 连接起来生成最终的(无限)结果。当然,\code{triples} 的调用者永远无法消费整个列表,但这并不重要,因为 Haskell 是惰性的。

通常需要一组三重嵌套循环的问题在列表 monad 和 \code{do} 语法的帮助下得到了极大的简化。如果这还不够,Haskell 允许你使用列表推导进一步简化此代码:

\src{snippet05}
这只是列表 monad 的进一步语法糖(严格来说,\code{MonadPlus})。

你可能会在其他函数式或命令式语言中看到类似的构造,它们伪装成生成器和协程。

\subsection{只读状态(Read-Only State)}

一个具有只读访问某些外部状态或环境的函数总是可以替换为一个以该环境作为额外参数的函数。一个纯函数 \code{(a, e) -> b}(其中 \code{e} 是环境的类型)乍一看不像是 Kleisli 箭头。但只要我们将它柯里化为 \code{a -> (e -> b)},我们就会认识到这种装饰是我们老朋友 reader 函子:

\src{snippet06}
你可以将返回 \code{Reader} 的函数解释为生成一个小的可执行文件:一个给定环境生成所需结果的动作。有一个辅助函数 \code{runReader} 来执行此类操作:

\src{snippet07}
它可以为不同的环境值生成不同的结果。

请注意,返回 \code{Reader} 的函数和 \code{Reader} 操作本身都是纯的。

要为 \code{Reader} monad 实现 bind,首先请注意,你必须生成一个接受环境 \code{e} 并生成 \code{b} 的函数:

\src{snippet08}
在 lambda 内部,我们可以执行操作 \code{ra} 生成一个 \code{a}:

\src{snippet09}
然后我们可以将 \code{a} 传递给继续体 \code{k} 以获得一个新的操作 \code{rb}:

\src{snippet10}
最后,我们可以使用环境 \code{e} 运行操作 \code{rb}:

\src{snippet11}
要实现 \code{return},我们创建一个忽略环境并返回不变值的操作。

将所有内容放在一起,经过一些简化后,我们得到以下定义:

\src{snippet12}

\subsection{只写状态(Write-Only State)}

这只是我们最初的日志记录示例。装饰由 \code{Writer} 函子给出:

\src{snippet13}
为了完整性,还有一个简单的辅助函数 \code{runWriter} 用于解包数据构造器:

\src{snippet14}
如前所见,为了使 \code{Writer} 可组合,\code{w} 必须是一个 monoid。这是使用 bind 运算符编写的 \code{Writer} monad 实例:

\src{snippet15}

\subsection{状态(State)}

具有读/写状态访问权限的函数结合了 \code{Reader} 和 \code{Writer} 的装饰。你可以将它们视为以状态作为额外参数的纯函数,并生成一个对值/状态的结果对:\code{(a, s) -> (b, s)}。在柯里化后,我们将它们转化为 Kleisli 箭头形式 \code{a -> (s -> (b, s))},装饰抽象在 \code{State} 函子中:

\src{snippet16}
再次强调,我们可以将 Kleisli 箭头视为返回一个动作,可以使用辅助函数执行此动作:

\src{snippet17}
不同的初始状态不仅可能生成不同的结果,还可能生成不同的最终状态。

\code{State} monad 的 bind 实现与 \code{Reader} monad 的实现非常相似,只不过在每一步都要注意传递正确的状态:

\src{snippet18}
这是完整的实例:

\src{snippet19}
还有两个辅助 Kleisli 箭头可以用来操作状态。其中一个检索状态进行检查:

\src{snippet20}
另一个则用一个全新的状态替换它:

\src{snippet21}

\subsection{异常(Exceptions)}

抛出异常的命令式函数实际上是一个部分函数——这是一个不适用于某些参数值的函数。使用纯全函数来实现异常的最简单方法是使用 \code{Maybe} 函子。部分函数扩展为一个总函数,在合理的情况下返回 \code{Just a},而在不合理的情况下返回 \code{Nothing}。如果我们还想返回一些关于失败原因的信息,我们可以改用 \code{Either} 函子(第一个类型固定,例如为 \code{String})。

这是 \code{Maybe} 的 \code{Monad} 实例:

\src{snippet22}
请注意,\code{Maybe} 的 monadic 组合正确地在检测到错误时短路计算(不调用继续体 \code{k})。这是我们期望从异常中得到的行为。

\subsection{Continuations}

这是你在面试后可能遇到的情况:“不要打给我们,我们会打给你!” 你不再直接得到答案,而是需要提供一个处理函数,在结果可用时调用它。这种编程风格在结果尚未确定时特别有用,例如因为它正在由另一个线程评估或从远程网站获取。Kleisli 箭头在这种情况下返回一个接受处理函数的函数,该函数表示“剩余的计算”:

\src{snippet23}
处理函数 \code{a -> r} 在最终调用时生成类型为 \code{r} 的结果,并在最后返回该结果。continuation 是由结果类型参数化的。(实际上,这通常是某种状态指示器。)

还有一个辅助函数用于执行 Kleisli 箭头返回的操作。它接受处理函数并将其传递给 continuation:

\src{snippet24}
continuations 的组合非常困难,因此通过 monad 处理它们,特别是通过 \code{do} 语法,具有极大的优势。

让我们研究一下 bind 的实现。首先让我们看一下精简后的签名:

\src{snippet25}
我们的目标是创建一个函数,该函数接受处理函数 \code{(b -> r)} 并生成结果 \code{r}。所以这是我们的起点:

\src{snippet26}
在 lambda 内,我们希望使用适当的处理函数调用函数 \code{ka},该处理函数表示剩余的计算。我们将这个处理函数实现为一个 lambda:

\src{snippet27}
在这种情况下,剩余的计算包括首先调用 \code{kab} 和 \code{a},然后将 \code{hb} 传递给结果操作 \code{kb}:

\src{snippet28}
如你所见,continuations 是从内到外组合的。最终的处理函数 \code{hb} 是从计算的最内层调用的。这是完整的实例:

\src{snippet29}

\subsection{交互式输入(Interactive Input)}

这是最棘手的问题,也是许多困惑的来源。显然,一个像 \code{getChar} 这样的函数,如果它返回键盘上输入的字符,就不能是纯的。但是,如果它返回字符放在容器中呢?只要没有办法从这个容器中提取字符,我们就可以声称这个函数是纯的。每次你调用 \code{getChar} 它都会返回完全相同的容器。从概念上讲,这个容器将包含所有可能字符的叠加态。

如果你熟悉量子力学,你应该能够理解这个类比。它就像装有 Schrödinger 的猫的盒子——只不过没有办法打开或窥视盒子。该盒子使用特殊的内置 \code{IO} 函子定义。在我们的例子中,\code{getChar} 可以声明为一个 Kleisli 箭头:

\src{snippet30}
(实际上,由于从单位类型的函数等效于选择返回类型的一个值,\code{getChar} 的声明简化为 \code{getChar :: IO Char}。)

作为一个函子,\code{IO} 允许你使用 \code{fmap} 操作其内容。并且,作为一个函子,它可以存储任何类型的内容,不仅仅是字符。当你考虑到在 Haskell 中,\code{IO} 是一个 monad 时,这种方法的实际效用就显现出来了。这意味着你可以组合生成 \code{IO} 对象的 Kleisli 箭头。

你可能认为 Kleisli 组合允许你窥探 \code{IO} 对象的内容(如果我们继续量子类比,就会“塌缩波函数”)。确实,你可以将 \code{getChar} 与另一个 Kleisli 箭头组合,该箭头接受一个字符,例如将其转换为整数。问题在于第二个 Kleisli 箭头只能以 \code{(IO\ Int)} 的形式返回该整数。再一次,你最终会得到所有可能整数的叠加态。以此类推。Schrödinger 的猫永远不会出袋。一旦你进入 \code{IO} monad,就没有办法退出它。对于 \code{IO} monad 没有 \code{runState} 或 \code{runReader} 的等效物。没有 \code{runIO}!

那么你能用 Kleisli 箭头的结果 \code{IO} 对象做什么,除了与另一个 Kleisli 箭头组合?好吧,你可以从 \code{main} 返回它。在 Haskell 中,\code{main} 的签名为:

\src{snippet31}
你可以将其视为 Kleisli 箭头:

\src{snippet32}
从这个角度来看,Haskell 程序只是 \code{IO} monad 中的一个大 Kleisli 箭头。你可以使用 monadic 组合从更小的 Kleisli 箭头中组合它。如何处理生成的 \code{IO} 对象(也称为 \code{IO} 操作)取决于运行时系统。

请注意,箭头本身是一个纯函数——一直都是纯函数。肮脏的工作则留给系统。当它最终执行从 \code{main} 返回的 \code{IO} 操作时,它会做各种肮脏的事情,例如读取用户输入、修改文件、打印讨厌的信息、格式化磁盘等等。Haskell 程序永远不会弄脏自己的手(嗯,除了它调用 \code{unsafePerformIO} 时,但这是另一个故事)。

当然,由于 Haskell 是惰性的,\code{main} 几乎立即返回,肮脏的工作会立即开始。正是在执行 \code{IO} 操作期间,纯计算的结果被请求并按需计算。因此,实际上,程序的执行是纯(Haskell)代码和肮脏(系统)代码的交错进行。

对于 \code{IO} monad,有一种更为离奇但在数学模型上非常合理的解释。它将整个宇宙视为程序中的一个对象。注意,概念上,命令式模型将宇宙视为一个外部的全局对象,因此通过与该对象交互,执行 I/O 的过程具有副作用。它们可以读取和修改宇宙的状态。

我们已经知道如何在函数式编程中处理状态——我们使用 state monad。然而,宇宙的状态不像简单状态那样可以使用标准数据结构轻松描述。但是,只要我们从不直接与它交互,我们也不必这样做。只要我们假设存在一种类型 \code{RealWorld},并通过一些宇宙工程奇迹,运行时能够提供这种类型的对象就足够了。\code{IO} 操作只是一个函数:

\src{snippet33}
或者,在 \code{State} monad 的术语中:

\src{snippet34}
不过,\code{IO} monad 的 \code{>=>} 和 \code{return} 必须内置于语言中。

\subsection{交互式输出(Interactive Output)}

相同的 \code{IO} monad 用于封装交互式输出。 \code{RealWorld} 应该包含所有输出设备。你可能想知道,为什么我们不能直接从 Haskell 调用输出函数,并假装它们什么都不做。例如,为什么我们有:

\src{snippet35}
而不是更简单的:

\src{snippet36}
有两个原因:Haskell 是惰性的,所以它永远不会调用一个其输出——在此处为单位对象——未被用于任何用途的函数。而且,即使它不是惰性的,它仍然可以自由更改这些调用的顺序,从而使输出混乱。强制 Haskell 中两个函数顺序执行的唯一方法是通过数据依赖性。一个函数的输入必须依赖于另一个函数的输出。在 \code{IO} 操作之间传递 \code{RealWorld} 强制执行了顺序。

从概念上讲,在这个程序中:

\src{snippet37}
打印“World!”的操作接收到的是一个宇宙,其中“Hello ”已经显示在屏幕上。它输出一个新的宇宙,其中“Hello World!”显示在屏幕上。

\section{结论(Conclusion)}

当然,我只是浅尝了 monadic 编程的表面。Monads 不仅用纯函数完成了在命令式编程中通常用副作用完成的工作,而且它们还做到了高度的控制和类型安全。不过,它们也不是没有缺点的。关于 monads 的主要抱怨是它们不容易相互组合。当然,你可以使用 monad transformer 库来组合大多数基本 monads。创建一个结合状态和异常的 monad 堆栈相对容易,但没有公式可以将任意 monads 叠加在一起。
