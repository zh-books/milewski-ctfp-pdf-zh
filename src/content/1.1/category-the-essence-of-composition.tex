% !TEX root = ../../ctfp-print.tex

\lettrine[lhang=0.17]{范畴}{category} 是一个令人难以置信的简单概念。
一个范畴由\newterm{对象}(objects)和在它们之间的\newterm{箭头}(arrows)组成。因此,范畴非常容易用图形表示。一个对象可以画成一个圆或一个点,而一个箭头\ldots{} 就是一个箭头。(为了增加一些变化,我偶尔会把对象画成小猪,把箭头画成烟火。)但范畴的本质是\emph{组合}(composition)。或者,如果你愿意,组合的本质就是一个范畴。箭头是可以组合的,所以如果你有一个从对象 $A$ 到对象 $B$ 的箭头,以及另一个从对象 $B$ 到对象 $C$ 的箭头,那么一定存在一个箭头——它们的组合——从 $A$ 到 $C$。

\begin{figure}
  \centering
  \includegraphics[width=0.8\textwidth]{images/img_1330.jpg}
  \caption{在一个范畴中,如果有一个箭头从 $A$ 到 $B$,并且有另一个箭头从 $B$ 到 $C$,那么必然存在一个直接从 $A$ 到 $C$ 的箭头,这就是它们的组合。这个图示并不是一个完整的范畴,因为它缺少了恒等态射(identity morphisms),稍后会提到。}
\end{figure}

\section{箭头作为函数 (Arrows as Functions)}

这是不是已经太过抽象了?别灰心。我们来谈些具体的内容。将箭头(arrows),也称为\newterm{态射}(morphisms),视为函数。你有一个函数 $f$,它接受一个类型为 $A$ 的参数,并返回一个类型为 $B$ 的值。你还有另一个函数 $g$,它接受一个 $B$ 并返回一个 $C$。你可以通过将 $f$ 的结果传递给 $g$ 来组合它们。这样你就定义了一个新函数,它接受一个 $A$ 并返回一个 $C$。

在数学中,这样的组合用函数之间的一个小圆圈来表示:$g \circ f$。请注意,组合的顺序是从右到左的。对于某些人来说,这可能有些令人困惑。你可能熟悉 Unix 中的管道符号,如:

\begin{snip}{text}
  lsof | grep Chrome
\end{snip}
或者在 F\# 中的尖括号 \code{>>},它们都是从左到右进行操作的。但是在数学和 Haskell 中,函数的组合是从右到左的。如果你将 $g \circ f$ 理解为“f 之后的 g”,这会有所帮助。

让我们通过写一些 C 代码来使这一点更加明确。我们有一个函数 \code{f},它接受一个类型为 \code{A} 的参数,并返回一个类型为 \code{B} 的值:

\begin{snip}{text}
  B f(A a);
\end{snip}
还有另一个函数:

\begin{snip}{text}
  C g(B b);
\end{snip}
它们的组合是:

\begin{snip}{text}
  C g_after_f(A a)
    {
    return g(f(a));
  }
\end{snip}
这里,你再次看到从右到左的组合:\code{g(f(a))};这次是在 C 语言中。

我希望我能告诉你,在 C++ 标准库中有一个模板可以接受两个函数并返回它们的组合,但实际上没有。所以我们来试试 Haskell。以下是从 A 到 B 的函数声明:

\src{snippet01}
类似地:

\src{snippet02}
它们的组合是:

\src{snippet03}
一旦你看到 Haskell 中的这些东西有多么简单,无法在 C++ 中表达简单的函数概念就有点尴尬了。事实上,Haskell 允许你使用 Unicode 字符,所以你可以这样写组合:
% 不要“数学化”此代码块
\begin{snip}{text}
  g ◦ f
\end{snip}

你甚至可以使用 Unicode 的双冒号和箭头:
% 不要“数学化”此代码块
\begin{snipv}
  f \ensuremath{\Colon} A → B
\end{snipv}
所以这是第一课 Haskell:双冒号表示“类型为\ldots{}”。一个函数类型是通过在两个类型之间插入一个箭头创建的。你通过在两个函数之间插入一个点(或 Unicode 圆圈)来组合它们。

\section{组合的性质 (Properties of Composition)}

在任何范畴中,组合必须满足两个极为重要的性质。

\begin{enumerate}
  \item
  组合是结合的。如果你有三个可以组合的态射,$f$,$g$ 和 $h$,(即它们的对象端对端匹配),你在组合它们时不需要括号。在数学符号中,这表示为:
  \[h \circ (g \circ f) = (h \circ g) \circ f = h \circ g \circ f\]
  在(伪)Haskell 中:

  \src{snippet04}[b]
  (我说“伪”,因为函数的等号并没有定义。)

  当处理函数时,结合性是显而易见的,但在其他范畴中可能并不明显。

  \item
  对于每个对象 $A$,都有一个箭头,它是组合的单位。这个箭头从对象自身回到自身。作为组合的单位意味着,当与任何从 $A$ 开始或结束于 $A$ 的箭头组合时,它分别返回相同的箭头。对象 $A$ 的单位箭头被称为 $\idarrow[A]$($A$ 的\newterm{恒等}(identity))。在数学符号中,如果 $f$ 从 $A$ 到 $B$,那么
  \[f \circ \idarrow[A] = f\]
  并且
  \[\idarrow[B] \circ f = f\]
\end{enumerate}
在处理函数时,恒等箭头实现为恒等函数(identity function),它仅返回其参数。这个实现对于每种类型都是相同的,这意味着该函数是普遍多态的。在 C++ 中,我们可以将其定义为一个模板:

\begin{snip}{cpp}
  template<class T> T id(T x) { return x; }
\end{snip}
当然,在 C++ 中,事情没有那么简单,因为你不仅要考虑你传递的是什么,还要考虑如何传递(即按值、按引用、按 const 引用、按移动等)。

在 Haskell 中,恒等函数是标准库(Prelude)的一部分。以下是其声明和定义:

\src{snippet05}
正如你所见,在 Haskell 中,多态函数非常简单。在声明中,你只需用类型变量替换类型。诀窍在于:具体类型的名称总是以大写字母开头,类型变量的名称以小写字母开头。所以这里的 \code{a} 代表所有类型。

Haskell 函数定义由函数名和形式参数组成——这里只是一个 \code{x}。函数体在等号后面。对于新手来说,这种简洁性常常令人震惊,但你很快就会发现它是非常合理的。函数定义和函数调用是函数式编程的基础,因此它们的语法被简化到最小。不仅参数列表没有括号,参数之间也没有逗号(稍后你会看到,当我们定义多参数函数时)。

函数体总是一个表达式——函数中没有语句。函数的结果就是这个表达式——这里只是 \code{x}。

这就是我们的第二课 Haskell。

恒等条件可以写成(再次在伪 Haskell 中):

\src{snippet06}
你可能会问自己:为什么有人会在意恒等函数——一个什么都不做的函数?再想想,为什么我们要在意数字零?零是无的符号。古罗马人的数字系统中没有零,他们依然能够建造出卓越的道路和水渠,其中一些至今仍在使用。

像零或 $\id$ 这样的中立值在处理符号变量时非常有用。这就是为什么古罗马人在代数方面并不擅长,而阿拉伯人和波斯人,由于熟悉零的概念,则擅长代数。所以恒等函数在作为高阶函数的参数或返回值时非常方便。高阶函数使得函数的符号操作成为可能。它们是函数的代数。

总结:一个范畴由对象和箭头(态射)组成。箭头可以组合,组合是结合的。每个对象都有一个恒等箭头,作为组合下的单位。

\section{组合是编程的本质 (Composition is the Essence of Programming)}

函数式程序员有一种独特的方式来处理问题。他们首先会问一些非常禅宗式的问题。例如,当设计一个交互式程序时,他们会问:什么是交互?当实现康威的生命游戏时,他们可能会思考生命的意义。在这种精神下,我要问:什么是编程?在最基本的层面上,编程是告诉计算机该做什么。“取内存地址 x 的内容,并将其添加到寄存器 EAX 的内容中。”但即使我们用汇编语言编程,我们给计算机的指令也是更有意义的表达。我们正在解决一个非平凡的问题(如果是平凡的问题,我们就不需要计算机的帮助了)。我们如何解决问题?我们将大问题分解为小问题。如果小问题仍然太大,我们进一步分解它们,依此类推。最后,我们编写代码来解决所有的小问题。然后,编程的本质就来了:我们将这些代码片段组合起来,形成解决更大问题的方案。如果不能将这些部分重新组合在一起,分解就毫无意义。

这种分层分解和重组的过程并不是计算机强加给我们的。它反映了人类思维的局限性。我们的大脑一次只能处理少量的概念。心理学中被引用最多的论文之一,
\urlref{http://en.wikipedia.org/wiki/The_Magical_Number_Seven,_Plus_or_Minus_Two}{《神奇的七,正负二》},假设我们的大脑只能容纳 $7 \pm 2$ 个“块”信息。我们对人类短期记忆的理解细节可能在变化,但我们可以肯定的是,它是有限的。关键是我们无法处理一锅杂物或代码拼盘。我们需要结构,不是因为结构良好的程序看起来赏心悦目,而是因为否则我们的大脑无法有效地处理它们。我们经常形容某段代码优雅或美丽,但我们真正的意思是它易于我们有限的人类大脑处理。优雅的代码创造了正好大小的块,数量正好适合我们的精神消化系统来吸收它们。

那么,程序组合的合适块是什么?它们的表面积必须比它们的体积增长得慢。(我喜欢这个类比,因为几何物体的表面积随着其大小的平方增长——比体积增长得慢,体积随着其大小的立方增长。)表面积是我们需要的信息,用以组合块。体积是我们需要的信息,用以实现它们。这个想法是,一旦实现了一个块,我们就可以忘记其实现的细节,并专注于它如何与其他块交互。在面向对象编程中,表面就是对象的类声明或其抽象接口。在函数式编程中,它就是函数的声明。(我稍微简化了一些,但这就是要点。)

范畴论在这一点上是极端的,因为它积极地阻止我们查看对象的内部。范畴论中的一个对象是一个抽象的朦胧实体。你唯一能知道的就是它如何与其他对象关联——如何用箭头连接它们。这就是互联网搜索引擎通过分析进出链接来对网站进行排名的方式(除非它们作弊)。在面向对象编程中,一个理想化的对象只能通过其抽象接口可见(纯表面,无体积),方法扮演箭头的角色。当你必须深入研究对象的实现以理解如何将其与其他对象组合时,你就失去了编程范式的优势。

\section{挑战 (Challenges)}

\begin{enumerate}
  \tightlist
  \item
  尽你所能,在你最喜欢的语言中实现恒等函数(identity function)(如果你最喜欢的语言是 Haskell,请在你第二喜欢的语言中实现)。
  \item
  在你最喜欢的语言中实现组合函数(composition function)。它接受两个函数作为参数,并返回它们的组合。
  \item
  编写一个程序,尝试测试你的组合函数是否遵守恒等性。
  \item
  在任何意义上,万维网(world-wide web)是一个范畴吗?链接是态射吗?
  \item
  Facebook 是一个范畴吗?人是对象,友谊是态射?
  \item
  何时一个有向图是一个范畴?
\end{enumerate}
