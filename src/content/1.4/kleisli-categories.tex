% !TEX root = ../../ctfp-print.tex

\lettrine[lhang=0.17]{你}{已经看到了如何将} 类型和纯函数建模为范畴。我还提到过在范畴论中有一种方法可以建模副作用(side effects)或非纯函数。让我们看看这样一个例子:记录或跟踪执行过程的函数。在命令式语言中,这可能会通过修改某些全局状态来实现,例如:

\begin{snip}{cpp}
  string logger;

  bool negate(bool b) {
    logger += "Not so! ";
    return !b;
  }
\end{snip}
你知道这不是一个纯函数,因为它的记忆化版本将无法生成日志。此函数具有\newterm{副作用}。

在现代编程中,我们尽量避免全局可变状态——仅仅是因为并发的复杂性。而且你永远不会在库中编写这样的代码。

幸运的是,我们可以让这个函数变成纯函数。你只需显式传递日志,输入和输出日志。让我们通过添加一个字符串参数,并将常规输出与包含更新日志的字符串配对来实现这一点:

\begin{snip}{cpp}
  pair<bool, string> negate(bool b, string logger) {
    return make_pair(!b, logger + "Not so! ");
  }
\end{snip}
这个函数是纯函数,没有副作用,每次用相同的参数调用时都会返回相同的对(pair),并且如果需要可以将其记忆化。然而,考虑到日志的累积性质,你将不得不记忆化所有可能导致给定调用的历史记录。将会有一个单独的记忆条目用于:

\begin{snip}{cpp}
  negate(true, "It was the best of times. ");
\end{snip}
以及

\begin{snip}{cpp}
  negate(true, "It was the worst of times. ");
\end{snip}
等等。

这对于库函数来说也不是一个很好的接口。调用者可以选择忽略返回类型中的字符串,因此这不是一个巨大的负担;但它们被迫传递一个字符串作为输入,这可能会带来不便。

有没有一种更不显眼的方式来实现相同的效果?有没有办法分离关注点?在这个简单的例子中,函数 \code{negate} 的主要目的是将一个布尔值转换为另一个布尔值。记录日志是次要的。当然,记录的消息是函数特有的,但将消息聚合成一个连续日志的任务是一个单独的关注点。我们仍然希望函数生成一个字符串,但我们希望解除它生成日志的负担。因此,这里有一个折衷的解决方案:

\begin{snip}{cpp}
  pair<bool, string> negate(bool b) {
    return make_pair(!b, "Not so! ");
  }
\end{snip}
这个想法是,日志将在函数调用之间被聚合。

为了看看如何做到这一点,让我们切换到一个稍微现实一点的例子。我们有一个从字符串到字符串的函数,它将小写字符转换为大写字符:

\begin{snip}{cpp}
  string toUpper(string s) {
    string result;
    int (*toupperp)(int) = &toupper; // toupper is overloaded
    transform(begin(s), end(s), back_inserter(result), toupperp);
    return result;
  }
\end{snip}
还有另一个函数,它将字符串拆分为字符串向量,按空白边界拆分:

\begin{snip}{cpp}
  vector<string> toWords(string s) {
    return words(s);
  }
\end{snip}
实际工作在辅助函数 \code{words} 中完成:

\begin{snip}{cpp}
  vector<string> words(string s) {
    vector<string> result{""};
    for (auto i = begin(s); i != end(s); ++i)
      {
      if (isspace(*i))
      result.push_back("");
      else
      result.back() += *i;
    }
    return result;
  }
\end{snip}
我们希望修改函数 \code{toUpper} 和 \code{toWords},使它们能够在常规返回值的基础上附加一条消息字符串。

\begin{figure}[H]
  \centering
  \includegraphics[width=0.3\textwidth]{images/piggyback.jpg}
\end{figure}
\noindent
我们将“修饰”这些函数的返回值。让我们通过定义一个模板 \code{Writer} 来以通用的方式做到这一点,该模板封装了一个对,其第一个组件是任意类型 \code{A} 的值,第二个组件是一个字符串:

\begin{snip}{cpp}
  template<class A>
  using Writer = pair<A, string>;
\end{snip}
以下是修饰后的函数:

\begin{snip}{cpp}
  Writer<string> toUpper(string s) {
    string result;
    int (*toupperp)(int) = &toupper;
    transform(begin(s), end(s), back_inserter(result), toupperp);
    return make_pair(result, "toUpper ");
  }

  Writer<vector<string>> toWords(string s) {
    return make_pair(words(s), "toWords ");
  }
\end{snip}
我们希望将这两个函数组合成另一个修饰过的函数,该函数将字符串转换为大写并将其拆分为单词,同时生成这些操作的日志。以下是我们可能的实现方式:

\begin{snip}{cpp}
  Writer<vector<string>> process(string s) {
    auto p1 = toUpper(s);
    auto p2 = toWords(p1.first);
    return make_pair(p2.first, p1.second + p2.second);
  }
\end{snip}
我们已经达到了目标:日志的聚合不再是各个函数的关注点。它们生成自己的消息,然后在外部将其连接成更大的日志。

现在想象一个完全以这种方式编写的程序。这是一个充满重复性、易出错的代码的噩梦。但我们是程序员。我们知道如何处理重复的代码:我们将其抽象化!然而,这不是你一般的抽象——我们必须抽象\newterm{函数组合}本身。但组合是范畴论的本质,因此在编写更多代码之前,让我们从范畴的角度分析这个问题。

\section{Writer 范畴 (The Writer Category)}

通过修饰一组函数的返回类型以附加一些额外的功能的想法证明是非常有用的。我们将看到更多的例子。起点是我们常规的类型和函数的范畴。我们将类型保留为对象,但将态射重新定义为修饰后的函数。

例如,假设我们要修饰从 \code{int} 到 \code{bool} 的函数 \code{isEven}。我们将其转换为一个态射,该态射由修饰后的函数表示。重要的一点是,这个态射仍然被认为是对象 \code{int} 和 \code{bool} 之间的箭头,尽管修饰后的函数返回的是一个对:

\begin{snip}{cpp}
  pair<bool, string> isEven(int n) {
    return make_pair(n % 2 == 0, "isEven ");
  }
\end{snip}
根据范畴的定律,我们应该能够将这个态射与另一个从对象 \code{bool} 到任意对象的态射组合。特别是,我们应该能够将其与我们之前的 \code{negate} 组合:

\begin{snip}{cpp}
  pair<bool, string> negate(bool b) {
    return make_pair(!b, "Not so! ");
  }
\end{snip}
显然,我们不能以组合常规函数的方式组合这两个态射,因为输入/输出不匹配。它们的组合应该看起来更像这样:

\begin{snip}{cpp}
  pair<bool, string> isOdd(int n) {
    pair<bool, string> p1 = isEven(n);
    pair<bool, string> p2 = negate(p1.first);
    return make_pair(p2.first, p1.second + p2.second);
  }
\end{snip}
所以这是我们正在构建的新范畴中两个态射的组合的配方:

\begin{enumerate}
  \tightlist
  \item
  执行与第一个态射对应的修饰函数
  \item
  提取结果对的第一个组件,并将其传递给与第二个态射对应的修饰函数
  \item
  连接第一个结果的第二个组件(字符串)和第二个结果的第二个组件(字符串)
  \item
  返回一个新对,将最终结果的第一个组件与连接后的字符串组合在一起。
\end{enumerate}

如果我们希望将这种组合抽象为 C++ 中的高阶函数,则必须使用由三个类型参数化的模板,这三个类型对应于我们范畴中的三个对象。它应接受根据我们的规则可以组合的两个修饰函数,并返回第三个修饰函数:

\begin{snip}{cpp}
  template<class A, class B, class C>
  function<Writer<C>(A)> compose(function<Writer<B>(A)> m1,
  function<Writer<C>(B)> m2)
    {
    return [m1, m2](A x) {
      auto p1 = m1(x);
      auto p2 = m2(p1.first);
      return make_pair(p2.first, p1.second + p2.second);
    };
  }
\end{snip}
现在我们可以回到前面的例子,使用这个新模板来实现 \code{toUpper} 和 \code{toWords} 的组合:

\begin{snip}{cpp}
  Writer<vector<string>> process(string s) {
    return compose<string, string, vector<string>>(toUpper, toWords)(s);
  }
\end{snip}
在向 \code{compose} 模板传递类型时仍有很多噪音。如果你有一个支持返回类型推导的泛型lambda函数的 C++14 兼容编译器,这种噪音可以避免(此代码的贡献归于 Eric Niebler):

\begin{snip}{cpp}
  auto const compose = [](auto m1, auto m2) {
    return [m1, m2](auto x) {
      auto p1 = m1(x);
      auto p2 = m2(p1.first);
      return make_pair(p2.first, p1.second + p2.second);
    };
  };
\end{snip}
在这个新定义中,\code{process} 的实现简化为:

\begin{snip}{cpp}
  Writer<vector<string>> process(string s) {
    return compose(toUpper, toWords)(s);
  }
\end{snip}
但我们还没有完成。我们已经在新范畴中定义了组合,但是恒等态射是什么?它们不是我们常规的恒等函数!它们必须是从类型 A 返回到类型 A 的态射,这意味着它们是如下形式的修饰函数:

\begin{snip}{cpp}
  Writer<A> identity(A);
\end{snip}
它们必须像组合的单位元一样运作。如果你看一下我们定义的组合,你会看到一个恒等态射应该传递其参数而不做任何更改,并且只向日志中贡献一个空字符串:

\begin{snip}{cpp}
  template<class A> Writer<A> identity(A x) {
    return make_pair(x, "");
  }
\end{snip}
你可以很容易地说服自己,我们刚刚定义的范畴确实是一个合法的范畴。特别是,我们的组合是平凡的结合的。如果你跟踪每个对的第一个组件所发生的事情,它只是一个常规函数组合,这是结合的。第二个组件正在连接,而连接也是结合的。

一个聪明的读者可能会注意到,很容易将此构造推广到任何幺半群,而不仅仅是字符串幺半群。我们可以在 \code{compose} 中使用 \code{mappend},在 \code{identity} 中使用 \code{mempty}(代替 \code{+} 和 \code{""})。实际上,我们没有理由仅限于记录字符串。一个好的库编写者应该能够识别出使库工作的最小约束——这里记录库的唯一要求是日志具有幺半群属性。

\section{Haskell 中的 Writer 范畴 (Writer in Haskell)}

在 Haskell 中做同样的事情更简洁一些,并且我们还得到了编译器的更多帮助。让我们从定义 \code{Writer} 类型开始:

\src{snippet01}
这里我只是在定义一个类型别名,相当于 C++ 中的 \code{typedef}(或 \code{using})。\code{Writer} 类型由类型变量 \code{a} 参数化,相当于 \code{a} 和 \code{String} 的对。对的语法是最简的:只是两个用逗号分隔的项目放在括号中。

我们的态射是从任意类型到某个 \code{Writer} 类型的函数:

\src{snippet02}
我们将声明组合为一个有趣的中缀运算符,有时称为“鱼”:

\src{snippet03}
这是一个有两个参数的函数,每个参数都是一个函数,并返回一个函数。第一个参数的类型是 \code{(a -> Writer b)},第二个参数是 \code{(b -> Writer c)},结果是 \code{(a -> Writer c)}。

以下是这个中缀运算符的定义——两个参数 \code{m1} 和 \code{m2} 出现在鱼符号的两边:

\src{snippet04}
结果是一个参数为 \code{x} 的 lambda 函数。lambda 写作反斜杠——将其视为希腊字母 $\lambda$ 的一条截肢腿。

\code{let} 表达式允许你声明辅助变量。这里调用 \code{m1} 的结果与一对变量 \code{(y, s1)} 进行模式匹配;调用 \code{m2} 的结果,使用第一个模式中的参数 \code{y},与 \code{(z, s2)} 进行匹配。

在 Haskell 中,通常匹配对而不是使用访问器,就像我们在 C++ 中所做的那样。除此之外,两种实现之间有一个相当直接的对应关系。

\code{let} 表达式的整体值在其 \code{in} 子句中指定:这里是一个对,其第一个组件是 \code{z},第二个组件是两个字符串 \code{s1++s2} 的连接。

我还将在我们的范畴中定义恒等态射,但由于某些原因,这些原因将在后面详细解释,我将其称为 \code{return}。

\src{snippet05}
为了完整性,我们来看一下 Haskell 版本的修饰函数 \code{upCase} 和 \code{toWords}:

\src{snippet06}
函数 \code{map} 对应于 C++ 中的 \code{transform}。它将字符函数 \code{toUpper} 应用于字符串 \code{s}。辅助函数 \code{words} 定义在标准 Prelude 库中。

最后,两个函数的组合使用鱼运算符完成:

\src{snippet07}

\section{Kleisli 范畴 (Kleisli Categories)}

你可能已经猜到我没有临时发明这个范畴。它是所谓的 Kleisli 范畴的一个例子——一个基于单子的范畴。我们还没有准备好讨论单子,但我想让你了解它们可以做什么。对于我们的有限目的,Kleisli 范畴的对象是底层编程语言的类型。从类型 $A$ 到类型 $B$ 的态射是从 $A$ 到使用特定修饰从 $B$ 派生出的类型的函数。每个 Kleisli 范畴定义了它自己的组合这些态射的方法,以及相对于该组合的恒等态射。(稍后我们将看到,不精确的术语“修饰”对应于范畴中的端函子概念。)

我在本章中用作范畴基础的特定单子称为\newterm{Writer 单子},它用于记录或跟踪函数的执行。它也是将效果嵌入纯计算中的更一般机制的一个例子。你之前已经看到,我们可以在集合范畴中建模编程语言的类型和函数(像往常一样,忽略底层)。在这里,我们将此模型扩展到一个略有不同的范畴,其中态射由修饰函数表示,并且它们的组合不仅仅是将一个函数的输出传递给另一个函数的输入。我们有了一个额外的自由度:组合本身。事实证明,这正是使得能够为在命令式语言中传统上使用副作用实现的程序提供简单指称语义的自由度。

\section{挑战 (Challenge)}

一个未定义其参数所有可能值的函数称为部分函数(partial function)。从数学意义上讲,它并不是真正的函数,因此它不符合标准的范畴模型。然而,它可以通过返回一个修饰类型 \code{optional} 的函数来表示:

\begin{snip}{cpp}
  template<class A> class optional {
    bool _isValid;
    A _value;
    public:
    optional()    : _isValid(false) {}
    optional(A v) : _isValid(true), _value(v) {}
    bool isValid() const { return _isValid; }
    A value() const { return _value; }
  };
\end{snip}
例如,下面是修饰函数 \code{safe\_root} 的实现:

\begin{snip}{cpp}
  optional<double> safe_root(double x) {
    if (x >= 0) return optional<double>{sqrt(x)};
    else return optional<double>{};
  }
\end{snip}
挑战如下:

\begin{enumerate}
  \tightlist
  \item
  构造部分函数的 Kleisli 范畴(定义组合和恒等态射)。
  \item
  实现修饰函数 \code{safe\_reciprocal},该函数返回其参数的有效倒数(reciprocal),如果它不为零。
  \item
  组合函数 \code{safe\_root} 和 \code{safe\_reciprocal} 以实现 \code{safe\_root\_reciprocal},该函数在可能的情况下计算 \code{sqrt(1/x)}。
\end{enumerate}
