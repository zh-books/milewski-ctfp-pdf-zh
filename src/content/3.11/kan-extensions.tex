% !TEX root = ../../ctfp-print.tex

\lettrine[lhang=0.17]{到}{目前为止,我们主要在处理}单一范畴(category)或一对范畴。在某些情况下,这有些过于限制。

例如,当我们在一个范畴 $\cat{C}$ 中定义极限(limit)时,我们引入了一个索引范畴(index category) $\cat{I}$ 作为模板,这个模板将构成我们圆锥体(cone)基础的模式。此时,引入另一个范畴(一个简单的范畴)来作为圆锥体顶点的模板是有意义的。然而,我们使用了从 $\cat{I}$ 到 $\cat{C}$ 的常量函子(constant functor)$\Delta_c$ 来解决这个问题。

现在是时候修正这种尴尬了。让我们使用三个范畴来定义极限。首先,我们从索引范畴 $\cat{I}$ 到范畴 $\cat{C}$ 的函子 $D$ 开始。这个函子选择了圆锥体的基础——即图函子(diagram functor)。

\begin{figure}[H]
  \centering
  \includegraphics[width=0.4\textwidth]{images/kan2.jpg}
\end{figure}

\noindent
新的添加是范畴 $\cat{1}$,它只包含一个对象(以及一个单一的恒等态射)。从 $\cat{I}$ 到这个范畴唯一可能的函子 $K$ 将所有对象映射到 $\cat{1}$ 中唯一的对象,并将所有态射映射到恒等态射。任何从 $\cat{1}$ 到 $\cat{C}$ 的函子 $F$ 选择了我们圆锥体的潜在顶点。

\begin{figure}[H]
  \centering
  \includegraphics[width=0.4\textwidth]{images/kan15.jpg}
\end{figure}

\noindent
圆锥体是从 $F \circ K$ 到 $D$ 的自然变换(natural transformation)$\varepsilon$。注意,$F \circ K$ 做的与我们最初的 $\Delta_c$ 是一样的事情。下图展示了这一变换。

\begin{figure}[H]
  \centering
  \includegraphics[width=0.4\textwidth]{images/kan3-e1492120491591.jpg}
\end{figure}

\noindent
我们现在可以定义一个普遍性质(universal property),来选择最“优”的函子 $F$。这个 $F$ 会将 $\cat{1}$ 映射到对象 $D$ 在范畴 $\cat{C}$ 中的极限上,并且自然变换 $\varepsilon$ 将提供相应的投影。这一普遍函子被称为函子 $D$ 沿着 $K$ 的右Kan扩展(right Kan extension),并用 $\Ran_{K}D$ 表示。

我们来表述一下这个普遍性质。假设我们有另一个圆锥体,即另一个函子 $F'$ 及其与 $F' \circ K$ 到 $D$ 的自然变换 $\varepsilon'$。

\begin{figure}[H]
  \centering
  \includegraphics[width=0.4\textwidth]{images/kan31-e1492120512209.jpg}
\end{figure}

\noindent
如果右Kan扩展 $F = \Ran_{K}D$ 存在,那么必定存在一个从 $F'$ 到它的唯一自然变换 $\sigma$,使得 $\varepsilon'$ 通过 $\varepsilon$ 分解,即:
\[\varepsilon' = \varepsilon\ .\ (\sigma \circ K)\]
这里,$\sigma \circ K$ 是两个自然变换的水平组合(其中一个是 $K$ 上的恒等自然变换)。然后将此变换与 $\varepsilon$ 垂直组合。

\begin{figure}[H]
  \centering
  \includegraphics[width=0.4\textwidth]{images/kan5.jpg}
\end{figure}

\noindent
在分量上,作用在范畴 $\cat{I}$ 中的对象 $i$ 上时,我们得到:
\[\varepsilon'_i = \varepsilon_i \circ \sigma_{K i}\]
在我们的例子中,$\sigma$ 只有一个对应于 $\cat{1}$ 中单一对象的分量。所以,这的确是从 $F'$ 定义的圆锥体顶点到 $\Ran_{K}D$ 定义的普遍圆锥体顶点的唯一态射。所需的交换条件正是极限定义所要求的。

但重要的是,我们可以将简单范畴 $\cat{1}$ 替换为任意范畴 $\cat{A}$,右Kan扩展的定义依然有效。

\section{右Kan扩展(Right Kan Extension)}

函子 $D \Colon \cat{I} \to \cat{C}$ 沿着函子 $K \Colon \cat{I} \to \cat{A}$ 的右Kan扩展是一个函子 $F \Colon \cat{A} \to \cat{C}$(记作 $\Ran_{K}D$),并伴随着一个自然变换
\[\varepsilon \Colon F \circ K \to D\]
使得对于任何其他函子 $F' \Colon \cat{A} \to \cat{C}$ 和自然变换
\[\varepsilon' \Colon F' \circ K \to D\]
都有一个唯一的自然变换
\[\sigma \Colon F' \to F\]
使得 $\varepsilon'$ 通过 $\varepsilon$ 分解:
\[\varepsilon' = \varepsilon\ .\ (\sigma \circ K)\]
这听起来有点复杂,但可以用下图来直观地表示:

\begin{figure}[H]
  \centering
  \includegraphics[width=0.4\textwidth]{images/kan7.jpg}
\end{figure}

\noindent
一个有趣的观点是,注意到在某种意义上,Kan扩展起到了“函子乘法”逆操作的作用。有些作者甚至使用 $D/K$ 来表示 $\Ran_{K}D$。确实,用这种表示法,右Kan扩展的定义 $\varepsilon$(也称为右Kan扩展的余单元(counit))看起来像是简单的消去:
\[\varepsilon \Colon D/K \circ K \to D\]
对Kan扩展还有另一种解释。考虑函子 $K$ 嵌入范畴 $\cat{I}$ 到 $\cat{A}$ 中。在最简单的情况下,$\cat{I}$ 可能只是 $\cat{A}$ 的一个子范畴。我们有一个函子 $D$ 将 $\cat{I}$ 映射到 $\cat{C}$。我们能否将 $D$ 扩展为定义在整个 $\cat{A}$ 上的函子 $F$?理想情况下,这样的扩展会使得 $F \circ K$ 与 $D$ 同构。换句话说,$F$ 会将 $D$ 的定义域扩展到 $\cat{A}$ 上。但通常来说,完全的同构往往过于苛刻,我们可以只要求一个方向的自然变换 $\varepsilon$ 从 $F \circ K$ 到 $D$(左Kan扩展取的是另一方向)。

\begin{figure}[H]
  \centering
  \includegraphics[width=0.4\textwidth]{images/kan6.jpg}
\end{figure}

\noindent
当然,当函子 $K$ 在对象上不单射或在同态集(hom-sets)上不忠实时,嵌入的概念就会失效,例如在极限的例子中。此时,Kan扩展会尽其所能地推测丢失的信息。

\section{Kan扩展作为伴随(Adjunction)}

现在假设右Kan扩展对于任何 $D$(和一个固定的 $K$)都存在。在这种情况下,$\Ran_{K}-$(用破折号替代 $D$)是一个从函子范畴 ${[}\cat{I}, \cat{C}{]}$ 到函子范畴 ${[}\cat{A}, \cat{C}{]}$ 的函子。事实证明,这个函子是前合成函子(precomposition functor) $- \circ K$ 的右伴随。后者将 ${[}\cat{A}, \cat{C}{]}$ 中的函子映射到 ${[}\cat{I}, \cat{C}{]}$ 中。这个伴随关系是:
\[[\cat{I}, \cat{C}](F' \circ K, D) \cong [\cat{A}, \cat{C}](F', \Ran_{K}D)\]
这只是对我们称为 $\varepsilon'$ 的自然变换对应唯一的自然变换 $\sigma$ 这一事实的重述。

\begin{figure}[H]
  \centering
  \includegraphics[width=0.4\textwidth]{images/kan92.jpg}
\end{figure}

\noindent
此外,如果我们选择范畴 $\cat{I}$ 与 $\cat{C}$ 相同,我们可以用恒等函子 $I_{\cat{C}}$ 替换 $D$。我们得到如下恒等式:
\[[\cat{C}, \cat{C}](F' \circ K, I_{\cat{C}}) \cong [\cat{A}, \cat{C}](F', \Ran_{K}I_{\cat{C}})\]
我们现在可以选择 $F'$ 与 $\Ran_{K}I_{\cat{C}}$ 相同。在这种情况下,右侧包含恒等自然变换,对应的左侧给出了以下自然变换:
\[\varepsilon \Colon \Ran_{K}I_{\cat{C}} \circ K \to I_{\cat{C}}\]
这非常类似于一个伴随的余单元:
\[\Ran_{K}I_{\cat{C}} \dashv K\]
确实,恒等函子 $I_{\cat{C}}$ 沿着一个函子 $K$ 的右Kan扩展可以用来计算 $K$ 的左伴随。为此,还需要一个条件:右Kan扩展必须被函子 $K$ 所保持。保持扩展意味着,如果我们计算与 $K$ 前合成后的函子的Kan扩展,我们应该得到与前合成原始Kan扩展的结果相同。在我们的例子中,这个条件简化为:
\[K \circ \Ran_{K}I_{\cat{C}} \cong \Ran_{K}K\]
注意,使用除以 $K$ 的表示法,伴随关系可以写为:
\[I/K \dashv K\]
这证实了我们对伴随描述某种逆操作的直觉。保持条件变为:
\[K \circ I/K \cong K/K\]
一个函子沿自身的右Kan扩展,$K/K$,被称为余密(codensity)单子(monad)。

伴随公式是一个重要的结果,因为正如我们很快会看到的,我们可以使用积(end)(或余积(coend))来计算Kan扩展,从而为我们提供了实际计算右(和左)伴随的方法。

\section{左Kan扩展(Left Kan Extension)}

有一个对偶的构造给出了左Kan扩展。为了建立一些直觉,我们可以从余极限(colimit)的定义开始,并将其重新构建为使用单例范畴 $\cat{1}$。我们使用函子 $D \Colon \cat{I} \to \cat{C}$ 形成基底,并使用函子 $F \Colon \cat{1} \to \cat{C}$ 选择顶点,来构建一个余锥体(cocone)。

\begin{figure}[H]
  \centering
  \includegraphics[width=0.4\textwidth]{images/kan81.jpg}
\end{figure}

\noindent
余锥体的侧边,即插入(injections),是从 $D$ 到 $F \circ K$ 的自然变换 $\eta$ 的分量。

\begin{figure}[H]
  \centering
  \includegraphics[width=0.4\textwidth]{images/kan10a.jpg}
\end{figure}

\noindent
余极限是普遍余锥体。所以对于任何其他函子 $F'$ 和自然变换
\[\eta' \Colon D \to F' \circ K\]

\begin{figure}[H]
  \centering
  \includegraphics[width=0.4\textwidth]{images/kan10b.jpg}
\end{figure}

\noindent
都有一个从 $F$ 到 $F'$ 的唯一自然变换 $\sigma$:

\begin{figure}[H]
  \centering
  \includegraphics[width=0.4\textwidth]{images/kan14.jpg}
\end{figure}

\noindent
使得:
\[\eta' = (\sigma \circ K)\ .\ \eta\]
如以下图所示:

\begin{figure}[H]
  \centering
  \includegraphics[width=0.4\textwidth]{images/kan112.jpg}
\end{figure}

\noindent
将单例范畴 $\cat{1}$ 替换为 $\cat{A}$,这个定义自然推广到左Kan扩展的定义,记作 $\Lan_{K}D$。

\begin{figure}[H]
  \centering
  \includegraphics[width=0.4\textwidth]{images/kan12.jpg}
\end{figure}

\noindent
自然变换:
\[\eta \Colon D \to \Lan_{K}D \circ K\]
被称为左Kan扩展的单元(unit)。

和之前一样,我们可以将自然变换 $\eta' = (\sigma \circ K)\ .\ \eta$ 的一对一对应关系重新表述为伴随:
\[[\cat{A}, \cat{C}](\Lan_{K}D, F') \cong [\cat{I}, \cat{C}](D, F' \circ K)\]
换句话说,左Kan扩展是左伴随,而右Kan扩展是前合成函子的右伴随。

就像恒等函子的右Kan扩展可以用来计算 $K$ 的左伴随一样,恒等函子的左Kan扩展最终成为 $K$ 的右伴随(其中 $\eta$ 是伴随的单元):
\[K \dashv \Lan_{K}I_{\cat{C}}\]
结合这两个结果,我们得到:
\[\Ran_{K}I_{\cat{C}} \dashv K \dashv \Lan_{K}I_{\cat{C}}\]

\section{Kan扩展作为积(Ends)}

Kan扩展的真正威力在于它们可以使用积(end)(和余积(coend))来计算。为了简单起见,我们将注意力限制在目标范畴 $\cat{C}$ 为 $\Set$ 的情况下,但这些公式可以推广到任何范畴。

让我们重新审视一个Kan扩展可以用来扩展函子的作用范围的想法。假设 $K$ 嵌入 $\cat{I}$ 到 $\cat{A}$ 中。函子 $D$ 将 $\cat{I}$ 映射到 $\Set$。我们可以简单地说,对于 $K$ 的像中的任何对象 $a$,即 $a = K i$,扩展后的函子将 $a$ 映射到 $D i$。问题是,如何处理那些在 $K$ 的像之外的 $\cat{A}$ 中的对象?想法是每个这样的对象都有可能通过大量的态射与 $K$ 的像中的每个对象连接在一起。一个函子必须保持这些态射的结构。一个对象 $a$ 到 $K$ 的像的态射的总和由同态函子(hom-functor)表征:
\[\cat{A}(a, K -)\]

\begin{figure}[H]
  \centering
  \includegraphics[width=0.4\textwidth]{images/kan13.jpg}
\end{figure}

\noindent
注意到这个同态函子是两个函子的合成:
\[\cat{A}(a, K -) = \cat{A}(a, -) \circ K\]
右Kan扩展是函子合成的右伴随:
\[[\cat{I}, \Set](F' \circ K, D) \cong [\cat{A}, \Set](F', \Ran_{K}D)\]
让我们看看当我们用同态函子替换 $F'$ 时会发生什么:
\[[\cat{I}, \Set](\cat{A}(a, -) \circ K, D) \cong [\cat{A}, \Set](\cat{A}(a, -), \Ran_{K}D)\]
然后将合成公式内联化:
\[[\cat{I}, \Set](\cat{A}(a, K -), D) \cong \Ran_{K}D a\]
我们现在可以将自然变换集表示为一个积,从而得到这个非常方便的右Kan扩展公式:
\[\Ran_{K}D a \cong \int_i \Set(\cat{A}(a, K i), D i)\]
左Kan扩展也有一个类似的用余积表示的公式:
\[\Lan_{K}D a = \int^i \cat{A}(K i, a)\times{}D i\]
为了证明这一点,我们将展示这确实是前合成函子的左伴随:
\[[\cat{A}, \Set](\Lan_{K}D, F') \cong [\cat{I}, \Set](D, F' \circ K)\]
让我们在左侧公式中代入我们的公式:
\[[\cat{A}, \Set](\int^i \cat{A}(K i, -)\times{}D i, F')\]
这是一组自然变换,所以它可以被重写为一个积:
\[\int_a \Set(\int^i \cat{A}(K i, a)\times{}D i, F' a)\]
利用同态函子的连续性,我们可以将余积替换为积:
\[\int_a \int_i \Set(\cat{A}(K i, a)\times{}D i, F' a)\]
我们可以使用乘积-指数(product-exponential)伴随:
\[\int_a \int_i \Set(\cat{A}(K i, a),\ (F' a)^{D i})\]
指数同构于相应的同态集:
\[\int_a \int_i \Set(\cat{A}(K i, a),\ \Set(D i, F' a))\]
有一个叫做Fubini定理的定理允许我们交换两个积:
\[\int_i \int_a \Set(\cat{A}(K i, a),\ \Set(D i, F' a))\]
内部积表示两个函子之间的自然变换集,所以我们可以使用Yoneda引理:
\[\int_i \Set(D i, F' (K i))\]
这确实是我们要证明的伴随右侧的自然变换集:
\[[\cat{I}, \Set](D, F' \circ K)\]
这些使用积、余积和Yoneda引理的计算对于“积的微积分”来说是非常典型的。

\section{Haskell中的Kan扩展}

Kan扩展的积/余积公式可以很容易地翻译成Haskell。让我们从右Kan扩展开始:
\[\Ran_{K}D a \cong \int_i \Set(\cat{A}(a, K i), D i)\]
我们用通用量词替代积,并用函数类型替代同态集:

\src{snippet01}
从这个定义来看,很明显 \code{Ran} 必须包含一个可以应用于函数的类型 \code{a},以及两个函子 \code{k} 和 \code{d} 之间的自然变换。例如,假设 \code{k} 是树函子(tree functor),\code{d} 是列表函子(list functor),你被给了一个 \code{Ran Tree {[}{]} String}。如果你传递给它一个函数:

\src{snippet02}
你将得到一个 \code{Int} 列表,等等。右Kan扩展将使用你的函数生成一棵树,然后将其重新打包成一个列表。例如,你可以传递一个解析器,该解析器从字符串生成解析树,你将得到一个对应于该树的深度优先遍历的列表。

右Kan扩展可以用来计算给定函子的左伴随,通过用恒等函子替换函子 \code{d}。这导致的结果是函子 \code{k} 的左伴随由以下类型的多态函数集表示:

\src{snippet03}
假设 \code{k} 是从幺半群(monoids)范畴的遗忘函子(forgetful functor)。通用量词将遍历所有幺半群。当然,在Haskell中我们无法表达幺半群的定律,但以下是一个生成的自由函子的相应近似(遗忘函子 \code{k} 是对对象的恒等):

\src{snippet04}
正如预期的那样,它生成自由幺半群,或者Haskell列表:

\src{snippet05}
左Kan扩展是一个余积:
\[\Lan_{K}D a = \int^i \cat{A}(K i, a)\times{}D i\]
因此,它翻译为一个存在量词。符号上表示为:

\begin{snip}{haskell}
  Lan k d a = exists i. (k i -> a, d i)
\end{snip}
这可以在Haskell中使用 \acronym{GADT} 进行编码,或者使用一个通用量词的数据构造器:

\src{snippet06}
这个数据结构的解释是,它包含一个函数,该函数接受某些未指定的 \code{i} 的容器并生成一个 \code{a}。它还包含了这些 \code{i} 的容器。由于你不知道 \code{i} 是什么,这个数据结构唯一能做的事情就是检索 \code{i} 的容器,将其重新打包成由函子 \code{k} 定义的容器,并调用该函数以获得 \code{a}。例如,如果 \code{d} 是一棵树,\code{k} 是一个列表,你可以序列化树,用生成的列表调用函数并得到一个 \code{a}。

左Kan扩展可以用来计算一个函子的右伴随。我们知道乘积函子的右伴随是指数函子,所以让我们尝试使用Kan扩展来实现它:

\src{snippet07}
这确实与函数类型同构,如以下一对函数所示:

\src{snippet08}
正如前面所描述的一般情况一样,我们执行了以下步骤:

\begin{enumerate}
  \tightlist
  \item
  检索到 \code{x} 的容器(在这里,它只是一个简单的恒等容器),以及函数 \code{f}。
  \item
  使用恒等函子和对函子对的自然变换,将容器重新打包。
  \item
  调用函数 \code{f}。
\end{enumerate}

\section{自由函子(Free Functor)}

Kan扩展的一个有趣应用是自由函子的构造。它解决了以下实际问题:假设你有一个类型构造器(type constructor)——即对象之间的映射。是否有可能基于这个类型构造器定义一个函子?换句话说,我们能否定义一个态射映射,从而将这个类型构造器扩展为一个完备的自函子(endofunctor)?

关键的观察是,一个类型构造器可以被描述为一个其定义域是离散范畴(discrete category)的函子。离散范畴除了恒等态射外没有其他态射。给定一个范畴 $\cat{C}$,我们总是可以通过简单地丢弃所有非恒等态射来构造一个离散范畴 $\cat{|C|}$。从 $\cat{|C|}$ 到 $\cat{C}$ 的函子 $F$ 是一个简单的对象映射,或者我们在Haskell中称为类型构造器。还有一个典型的函子 $J$ 将 $\cat{|C|}$ 嵌入到 $\cat{C}$ 中:它在对象和恒等态射上是恒等的。$F$ 沿 $J$ 的左Kan扩展(如果存在)就是一个从 $\cat{C}$ 到 $\cat{C}$ 的函子:
\[\Lan_{J}F a = \int^i \cat{C}(J i, a)\times{}F i\]
它被称为基于 $F$ 的自由函子。

在Haskell中,我们会这样写:

\src{snippet09}
确实,对于任何类型构造器 \code{f},\code{FreeF f} 是一个函子:

\src{snippet10}
如你所见,自由函子通过记录函数及其参数来模拟函数的提升(lifting)。它通过记录它们的组合来累积提升的函数。函子规则自动得到满足。这种构造在论文\urlref{http://okmij.org/ftp/Haskell/extensible/more.pdf}{Freer Monads, More Extensible Effects}中被使用。

或者,我们可以使用右Kan扩展来达到相同目的:

\src{snippet11}
很容易验证这确实是一个函子:

\src{snippet12}
