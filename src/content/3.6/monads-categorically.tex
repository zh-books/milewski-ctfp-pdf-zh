% !TEX root = ../../ctfp-print.tex

\lettrine[lhang=0.17]{如}{果你向程序员提到}单子(monads),你可能会最终讨论效应(effects)。对于数学家来说,单子是关于代数(algebras)的。我们稍后会讨论代数——它们在编程中起着重要作用——但首先我想给你一些关于它们与单子关系的直觉。现在,这有点像是空谈,但请耐心听我讲下去。

代数涉及到创建、操作和求值表达式。表达式是使用运算符构建的。考虑这个简单的表达式:
\[x^2 + 2 x + 1\]
这个表达式是使用变量如 $x$ 和常数如 $1$ 或 $2$ 以及运算符如加号或乘号构成的。作为程序员,我们通常将表达式视为树结构。

\begin{figure}[H]
  \centering
  \includegraphics[width=0.3\textwidth]{images/exptree.png}
\end{figure}

\noindent
树是一种容器,因此更普遍地说,表达式是用于存储变量的容器。在范畴论中,我们将容器表示为自函子(endofunctors)。如果我们将类型 $a$ 分配给变量 $x$,我们的表达式将具有类型 $m\ a$,其中 $m$ 是一个构建表达式树的自函子(非平凡的分支表达式通常使用递归定义的自函子来创建)。

对表达式可以执行的最常见操作是什么?是替换:用表达式替换变量。例如,在我们的例子中,我们可以用 $y - 1$ 替换 $X$,得到:
\[(y - 1)^2 + 2 (y - 1) + 1\]
这里发生了什么:我们获取了一个类型为 $m\ a$ 的表达式,并应用了一个类型为 $a \to m\ b$ 的变换($b$ 代表 $y$ 的类型)。结果是一个类型为 $m\ b$ 的表达式。让我详细说明:
\[m\ a \to (a \to m\ b) \to m\ b\]
是的,这是单子绑定(monadic bind)的签名。

这是一些动机性的介绍。现在让我们进入单子的数学定义。数学家使用的记号与程序员不同。他们更喜欢使用字母 $T$ 表示自函子,并使用希腊字母:$\mu$ 表示 \code{join},$\eta$ 表示 \code{return}。这两个都是多态函数,所以我们可以猜测它们对应于自然变换。

因此,在范畴论中,单子定义为一个自函子 $T$,配备一对自然变换 $\mu$ 和 $\eta$。

$\mu$ 是从函子 $T^2$ 到 $T$ 的自然变换。$T^2$ 简单来说就是自函子自身的复合 $T \circ T$(这种平方运算只能对自函子进行)。
\[\mu \Colon T^2 \to T\]
这个自然变换在对象 $a$ 处的分量是态射:
\[\mu_a \Colon T (T a) \to T a\]
在 $\Hask$ 中,这直接转化为我们对 \code{join} 的定义。

$\eta$ 是恒等函子 $I$ 和 $T$ 之间的自然变换:
\[\eta \Colon I \to T\]
考虑到 $I$ 在对象 $a$ 上的作用只是 $a$,因此 $\eta$ 的分量由态射给出:
\[\eta_a \Colon a \to T a\]
这直接转化为我们对 \code{return} 的定义。

这些自然变换必须满足一些附加的规律。一种理解方式是,这些规律允许我们为自函子 $T$ 定义一个克莱斯里范畴(Kleisli category)。记住,$a$ 和 $b$ 之间的克莱斯里箭头定义为态射 $a \to T b$。两个这样的箭头的组合(我将其写作带下标 $T$ 的圆圈)可以使用 $\mu$ 实现:
\[g \circ_T f = \mu_c \circ (T g) \circ f\]
其中
\begin{gather*}
  f \Colon a \to T b \\
  g \Colon b \to T c
\end{gather*}
这里 $T$ 作为一个函子,可以应用于态射 $g$。在 Haskell 记号中,这个公式可能更容易识别:

\src{snippet01}
或者,在组件中:

\src{snippet02}
在代数解释中,我们只是组合了两个连续的替换。

为了使克莱斯里箭头形成一个范畴,我们希望它们的组合是结合的,并且 $\eta_a$ 是 $a$ 上的恒等克莱斯里箭头。这个要求可以转化为 $\mu$ 和 $\eta$ 的单子定律。但是还有另一种推导这些定律的方法,使它们更像幺半群定律。事实上,$\mu$ 通常被称为“乘法”(multiplication),$\eta$ 被称为“单位”(unit)。

粗略地说,结合律规定,将 $T$ 的立方 $T^3$ 降至 $T$ 的两种方式必须得出相同的结果。两个单位律(左单位律和右单位律)规定,当 $\eta$ 应用于 $T$ 然后由 $\mu$ 归约时,我们回到了 $T$。

由于我们在组合自然变换和函子,因此这些有点棘手。因此,有必要复习一下横向组合。例如,$T^3$ 可以看作 $T$ 作用于 $T^2$ 之后的组合。我们可以将两个自然变换的横向组合应用于它:
\[I_T \circ \mu\]

\begin{figure}[H]
  \centering
  \includegraphics[width=0.4\textwidth]{images/assoc1.png}
\end{figure}

\noindent
并得到 $T \circ T$;然后可以通过应用 $\mu$ 将其进一步简化为 $T$。$I_T$ 是从 $T$ 到 $T$ 的恒等自然变换。在这种情况下,你会经常看到 $I_T \circ \mu$ 的横向组合记法缩写为 $T \circ \mu$。这种记法是明确的,因为将函子与自然变换组合没有意义,因此在这种上下文中 $T$ 必须表示 $I_T$。

\noindent
我们还可以在(自)函子范畴 ${[}\cat{C}, \cat{C}{]}$ 中绘制图表:

\begin{figure}[H]
  \centering
  \includegraphics[width=0.3\textwidth]{images/assoc2.png}
\end{figure}

\noindent
或者,我们可以将 $T^3$ 看作 $T^2 \circ T$ 的组合,并对其应用 $\mu \circ T$。结果也是 $T \circ T$,然后可以用 $\mu$ 归约为 $T$。我们要求这两条路径产生相同的结果。

\begin{figure}[H]
  \centering
  \includegraphics[width=0.3\textwidth]{images/assoc.png}
\end{figure}

\noindent
类似地,我们可以对 $T$ 之后的恒等函子的组合应用横向组合 $\eta \circ T$,得到 $T^2$,然后可以用 $\mu$ 归约。结果应该与我们直接将恒等自然变换应用于 \code{T} 的结果相同。同样的,$T \circ \eta$ 也是如此。

\begin{figure}[H]
  \centering
  \includegraphics[width=0.4\textwidth]{images/unitlawcomp-1.png}
\end{figure}

\noindent
你可以确信这些定律保证了克莱斯里箭头的组合确实满足范畴的定律。

单子和幺半群之间的相似性是显而易见的。我们有乘法 $\mu$,单位 $\eta$,结合律和单位律。但是我们对幺半群的定义过于狭窄,无法将单子描述为幺半群。因此,让我们将幺半群的概念推广。

\section{幺半群范畴(Monoidal Categories)}

让我们回到幺半群的传统定义。幺半群是一个集合,带有一个二元运算和一个称为单位元的特殊元素。在 Haskell 中,这可以用一个类型类来表达:

\src{snippet03}
二元运算 \code{mappend} 必须是结合的,并且具有单位性(即乘以单位元 \code{mempty} 不会改变任何元素)。

注意,在 Haskell 中,\code{mappend} 的定义是柯里化的。它可以解释为将 \code{m} 的每个元素映射到一个函数:

\src{snippet04}
正是这种解释导致了幺半群被定义为一个单对象范畴,其中的态射 \code{(m -> m)} 代表幺半群的元素。但由于柯里化是 Haskell 的内置特性,我们也可以从另一种乘法定义开始:

\src{snippet05}
在这里,笛卡尔积 \code{(m, m)} 成为待乘元素对的来源。

这个定义暗示了另一种推广的路径:用范畴积(categorical product)替换笛卡尔积。我们可以从一个在全局定义积的范畴开始,选择其中的一个对象 \code{m},并定义乘法为态射:
\[\mu \Colon m\times{}m \to m\]
但是我们有一个问题:在任意范畴中,我们无法窥探对象的内部,所以我们如何选择单位元呢?有一个技巧可以解决这个问题。记得元素选择等价于从单集合到某个集合的函数吗?在 Haskell 中,我们可以将 \code{mempty} 的定义替换为一个函数:

\src{snippet06}
单集合是 $\Set$ 中的终对象,因此将这个定义推广到任何具有终对象的范畴是自然的:
\[\eta \Colon t \to m\]
这使我们可以选择单位“元素”而无需讨论元素。

与我们之前将幺半群定义为单对象范畴不同,这里的幺半群定律不是自动满足的——我们必须强加它们。但是,为了表述它们,我们必须建立底层范畴积本身的幺半群结构。让我们首先回顾 Haskell 中的幺半群结构是如何工作的。

我们从结合性开始。在 Haskell 中,相应的等式定律是:

\src{snippet07}
在将其推广到其他范畴之前,我们必须将其重写为函数(态射)的相等性。我们必须将其抽象化,不再关注其对单个变量的作用——换句话说,我们必须使用点自由记法(point-free notation)。知道笛卡尔积是一个双函子,我们可以将左侧写为:

\src{snippet08}
右侧写为:

\src{snippet09}
这几乎是我们想要的。不幸的是,笛卡尔积并不是严格结合的——\code{(x, (y, z))} 不等同于 \code{((x, y), z)}——所以我们不能直接写点自由的:

\src{snippet10}
另一方面,这两种配对嵌套是同构的。存在一个可逆的函数,称为结合子(associator),可以在它们之间转换:

\src{snippet11}
借助结合子,我们可以为 \code{mu} 写出点自由的结合律:

\src{snippet12}
我们可以对单位律进行类似的处理,在新记法中,它们的形式为:

\src{snippet13}
它们可以重写为:

\src{snippet14}
同构 \code{lambda} 和 \code{rho} 分别称为左单位元(left unitor)和右单位元(right unitor)。它们见证了单位 \code{()} 是笛卡尔积的单位元,至多同构:

\src{snippet15}

\src{snippet16}
因此,单位律的点自由版本为:

\src{snippet17}
我们使用了底层笛卡尔积本身在类型范畴中充当幺半群乘法这一事实,得到了 \code{mu} 和 \code{eta} 的点自由幺半群定律。但请记住,笛卡尔积的结合性和单位律仅在同构意义上成立。

事实证明,这些定律可以推广到任何具有积和终对象的范畴。范畴积确实在同构意义上是结合的,终对象也是单位元,同样在同构意义上。结合子和两个单位元是自然同构。这些定律可以用交换图表示。

\begin{figure}[H]
  \centering
  \includegraphics[width=0.5\textwidth]{images/assocmon.png}
\end{figure}

\noindent
注意,由于积是一个双函子,它可以提升一对态射——在 Haskell 中,这是通过 \code{bimap} 完成的。

我们可以在此停下来,说我们可以在任何具有范畴积和终对象的范畴上定义幺半群。只要我们能选择一个对象 $m$ 以及两个满足幺半群定律的态射 $\mu$ 和 $\eta$,我们就有一个幺半群。但我们可以做得更好。我们不需要完整的范畴积来表述 $\mu$ 和 $\eta$ 的定律。回想一下积是通过一个使用投影的泛构造定义的。在我们的幺半群定律表述中并没有使用任何投影。

一种表现为积但并不是积的双函子称为\newterm{张量积}(tensor product),通常记作中缀运算符 $\otimes$。一般情况下,张量积的定义有点棘手,但我们不必担心这个。我们只列出它的性质——最重要的是它在同构意义上的结合性。

类似地,我们不需要对象 $t$ 是终对象。我们从未使用过它的终性质——即从任意对象到它的唯一态射的存在性。我们需要的是它能与张量积很好地配合工作。也就是说,我们希望它是张量积的单位元,同样是在同构意义上。让我们把所有这些拼在一起:

一个幺半群范畴是一个范畴 $\cat{C}$,它配备了一个称为张量积的双函子:
\[\otimes \Colon \cat{C}\times{}\cat{C} \to \cat{C}\]
以及一个特殊的对象 $i$,称为单位对象(unit object),并配有三个自然同构,分别称为结合子(associator)和左单位元(left unitor)以及右单位元(right unitor):
\begin{align*}
  \alpha_{a b c} & \Colon (a \otimes b) \otimes c \to a \otimes (b \otimes c) \\
  \lambda_a      & \Colon i \otimes a \to a                                   \\
  \rho_a         & \Colon a \otimes i \to a
\end{align*}
(还有一个用于简化四重张量积的相干性条件。)

重要的是,张量积描述了许多熟悉的双函子。特别是它适用于积、余积,并且正如我们将很快看到的,它适用于自函子的组合(以及一些更晦涩的积,如 Day convolution)。幺半群范畴在丰富范畴(enriched categories)的表述中起着至关重要的作用。

\section{幺半群范畴中的幺半群(Monoid in a Monoidal Category)}

现在我们准备好在更广泛的幺半群范畴中定义幺半群。我们首先选择一个对象 $m$。使用张量积,我们可以形成 $m$ 的幂。$m$ 的平方是 $m \otimes m$。有两种形成 $m$ 的立方的方法,但它们通过结合子是同构的。对于 $m$ 的更高幂也是如此(这就是我们需要相干性条件的地方)。为了形成幺半群,我们需要选择两个态射:
\begin{align*}
  \mu  & \Colon m \otimes m \to m \\
  \eta & \Colon i \to m
\end{align*}
其中 $i$ 是我们张量积的单位对象。

\begin{figure}[H]
  \centering
  \includegraphics[width=0.4\textwidth]{images/monoid-1.jpg}
\end{figure}

\noindent
这些态射必须满足结合律和单位律,它们可以用以下交换图表示:

\begin{figure}[H]
  \centering
  \includegraphics[width=0.5\textwidth]{images/assoctensor.jpg}
\end{figure}

\begin{figure}[H]
  \centering
  \includegraphics[width=0.5\textwidth]{images/unitmon.jpg}
\end{figure}

\noindent
注意,张量积必须是一个双函子,因为我们需要提升态射对来形成如 $\mu \otimes \id$ 或 $\eta \otimes \id$ 的积。这些图表只是我们之前对范畴积结果的直接推广。

\section{单子作为幺半群(Monads as Monoids)}

幺半群结构出现在意想不到的地方。其中一个地方是函子范畴。如果你稍微眯起眼睛,你可能会看到函子组合是一种乘法形式。问题在于并不是任意两个函子都可以组合——一个的目标范畴必须是另一个的源范畴。这只是态射组合的通常规则——我们知道,函子确实是范畴 $\Cat$ 中的态射。但是,就像自态射(从同一对象循环回的态射)总是可组合的一样,自函子也是如此。对于任意给定的范畴 $\cat{C}$,从 $\cat{C}$ 到 $\cat{C}$ 的自函子形成了函子范畴 ${[}\cat{C}, \cat{C}{]}$。它的对象是自函子,态射是它们之间的自然变换。我们可以从这个范畴中选择任意两个对象,比如自函子 $F$ 和 $G$,并生成第三个对象 $F \circ G$——一个它们的组合自函子。

自函子组合是张量积的一个好候选者吗?首先,我们必须确定它是一个双函子。它可以用于提升态射对——这里是自然变换吗?张量积的 \code{bimap} 的类似物的签名看起来像这样:
\[\mathit{bimap} \Colon (a \to b) \to (c \to d) \to (a \otimes c \to b \otimes d)\]
如果你将对象替换为自函子,将箭头替换为自然变换,并将张量积替换为组合,你会得到:
\[(F \to F') \to (G \to G') \to (F \circ G \to F' \circ G')\]
你可能会认识到这是横向组合的特例。

\begin{figure}[H]
  \centering
  \includegraphics[width=0.4\textwidth]{images/horizcomp.png}
\end{figure}

\noindent
我们还可以使用恒等自函子 $I$ 作为自函子组合的恒等元——我们新的张量积。此外,函子组合是结合的。事实上,结合律和单位律是严格的——不需要结合子或两个单位元。因此,自函子通过函子组合形成一个严格的幺半群范畴。

这个范畴中的幺半群是什么?它是一个对象——即自函子 $T$;以及两个态射——即自然变换:
\begin{gather*}
  \mu \Colon T \circ T \to T \\
  \eta \Colon I \to T
\end{gather*}
不仅如此,这里是幺半群定律:

\begin{figure}[H]
  \centering
  \includegraphics[width=0.4\textwidth]{images/assoc.png}
\end{figure}

\begin{figure}[H]
  \centering
  \includegraphics[width=0.5\textwidth]{images/unitlawcomp.png}
\end{figure}

\noindent
它们正是我们之前看到的单子定律。现在你明白了 Saunders Mac Lane 的那句著名的话:

\begin{quote}
  总而言之,单子只是自函子范畴中的幺半群。
\end{quote}
你可能在函数式编程会议上看到过这句话印在 T 恤上。

\section{从伴随子得出单子(Monads from Adjunctions)}

一个\hyperref[adjunctions]{伴随}关系\footnote{参见第 18 章关于\hyperref[adjunctions]{伴随}的内容。}
$L \dashv R$,是一个在两个范畴 $\cat{C}$ 和 $\cat{D}$ 之间来回的函子对。有两种组合它们的方法产生两个自函子,$R \circ L$ 和 $L \circ R$。根据伴随关系,这些自函子通过两个称为单位元(unit)和余单位元(counit)的自然变换与恒等函子相关联:
\begin{gather*}
  \eta \Colon I_{\cat{D}} \to R \circ L \\
  \varepsilon \Colon L \circ R \to I_{\cat{C}}
\end{gather*}
我们立即看到,伴随关系的单位元看起来就像单子的单位元。事实证明,自函子 $R \circ L$ 确实是一个单子。我们只需要定义与 $\eta$ 对应的 $\mu$。这是一个自函子的平方与自函子本身之间的自然变换,或者换句话说,是伴随函子的关系:
\[R \circ L \circ R \circ L \to R \circ L\]
事实上,我们可以使用余单位元来消除中间的 $L \circ R$。$\mu$ 的精确公式由横向组合给出:
\[\mu = R \circ \varepsilon \circ L\]
单子定律从伴随关系的单位元和余单位元的恒等式以及互换律中得出。

在 Haskell 中,我们不常见到从伴随关系派生的单子,因为伴随关系通常涉及两个范畴。然而,指数对象或函数对象的定义是一个例外。这里是形成这种伴随关系的两个自函子:
\begin{gather*}
  L z = z\times{}s \\
  R b = s \Rightarrow b
\end{gather*}
你可能会认出它们的组合是我们熟悉的状态单子:
\[R (L z) = s \Rightarrow (z\times{}s)\]
我们之前在 Haskell 中见过这个单子:

\src{snippet18}
让我们也将伴随关系翻译成 Haskell。左函子是积函子:

\src{snippet19}
右函子是读者函子:

\src{snippet20}
它们形成了伴随关系:

\src{snippet21}
你可以轻松地验证读者函子与积函子的组合确实等价于状态函子:

\src{snippet22}
如预期的那样,伴随关系的 \code{unit} 等价于状态单子的 \code{return} 函数。余单位元通过求值一个函数作用于其参数来起作用。这可以识别为函数 \code{runState} 的未柯里化版本:

\src{snippet23}
(未柯里化,因为在 \code{counit} 中它作用于一个对。)

现在我们可以将状态单子的 \code{join} 定义为自然变换 $\mu$ 的一个分量。为此,我们需要将三个自然变换进行横向组合:
\[\mu = R \circ \varepsilon \circ L\]
换句话说,我们需要通过读者函子的一个级别来偷偷地将余单位元 $\varepsilon$ 跨过。我们不能直接调用 \code{fmap},因为编译器会选择适用于 \code{State} 函子的版本,而不是 \code{Reader} 函子的版本。但请记住,读者函子的 \code{fmap} 只是左函数组合。因此,我们将直接使用函数组合。

我们首先需要剥离数据构造器 \code{State} 以暴露 \code{State} 函子内部的函数。这是通过 \code{runState} 完成的:

\src{snippet24}
然后我们将它与余单位元左组合,后者由 \code{uncurry runState} 定义。最后,我们将其重新包装回 \code{State} 数据构造器中:

\src{snippet25}
这确实是状态单子的 \code{join} 实现。

事实证明,不仅每个伴随关系都产生一个单子,反过来也成立:每个单子都可以分解为两个伴随函子的组合。然而,这种分解不是唯一的。

我们将在下一节讨论另一个自函子 $L \circ R$。
