% !TEX root = ../../ctfp-print.tex

\lettrine[lhang=0.17]{现}{在已经没有一个完美的地方} 可以结束关于范畴论的书籍。总有更多的内容可以学习。范畴论是一个庞大的学科。同时,很明显的是,相同的主题、概念和模式不断地反复出现。有一种说法认为,所有的概念都是Kan扩张(Kan extension),事实上,你可以用Kan扩张推导极限(limit)、余极限(colimit)、伴随(adjunction)、单子(monad)、Yoneda引理(Yoneda lemma)以及更多内容。范畴的概念本身在所有抽象层次上都出现,单子和单子的概念也是如此。那么,哪一个是最基础的呢?结果表明,它们彼此关联,相互引导,在一个永无止境的抽象循环中相互作用。我决定展示这些相互联系,可能是结束这本书的一个好方法。

\section{双范畴(Bicategories)}

范畴论最困难的方面之一就是不断转换视角。以集合范畴($\Set$)为例。我们习惯于用元素来定义集合。空集没有元素,单元素集有一个元素,两个集合的笛卡尔积(Cartesian product)是一个对的集合,等等。但是,当我们讨论集合范畴($\Set$)时,我让你忘记集合的内容,而是专注于它们之间的态射(箭头)。你可以偶尔窥视一下集合中的具体内容,以便理解范畴中的某个通用构造。终对象(terminal object)原来是一个只有一个元素的集合,等等。但这些只是一些验证检查。

函子(Functor)被定义为范畴之间的映射。自然地,可以将映射视为一个范畴中的态射。函子实际上是范畴范畴(category of categories)中的态射(箭头),如果我们希望避免关于大小的问题,应该说是小范畴(small categories)。通过将函子视为箭头,我们丢弃了关于它在范畴内部(对象和态射)作用的信息,就像我们在集合范畴中将函数视为箭头时丢弃了关于它对集合元素作用的信息一样。然而,两个范畴之间的函子也构成一个范畴。这一次,我要求你考虑在一个范畴中作为箭头的东西在另一个范畴中作为对象。在函子范畴中,函子是对象,自然变换是态射。我们发现,同样的东西在一个范畴中是箭头,在另一个范畴中是对象。物体作为名词和箭头作为动词的朴素观点并不成立。

我们可以尝试将两种视角合并为一种,这样我们就得到了$\cat{2}$-范畴(2-category)的概念,在这种范畴中,对象称为$0$-胞元(0-cells),态射称为$1$-胞元(1-cells),态射之间的态射称为$2$-胞元(2-cells)。

\begin{figure}[H]
  \centering
  \includegraphics[width=0.35\textwidth]{images/twocat.png}
  \caption{$0$-胞元 $a, b$;$1$-胞元 $f, g$;以及$2$-胞元 $\alpha$。}
\end{figure}

\noindent
范畴的范畴 $\Cat$ 是一个直接的例子。我们有范畴作为$0$-胞元,函子作为$1$-胞元,自然变换作为$2$-胞元。$\cat{2}$-范畴的规则告诉我们,任何两个$0$-胞元之间的$1$-胞元构成一个范畴(换句话说,$\cat{C}(a, b)$ 是一个同态范畴(hom-category),而不是同态集(hom-set))。这与我们早先的断言相吻合,即任何两个范畴之间的函子构成一个函子范畴。

特别地,从任何$0$-胞元回到自身的$1$-胞元也构成一个范畴,即同态范畴 $\cat{C}(a, a)$,但这个范畴具有更丰富的结构。$\cat{C}(a, a)$ 的成员可以看作是在范畴 $\cat{C}$ 中的箭头,或在 $\cat{C}(a, a)$ 中的对象。作为箭头,它们可以相互组合。但当我们将它们视为对象时,组合就变成了从一对对象到一个对象的映射。事实上,它看起来非常像一个积——确切地说,是一个张量积(tensor product)。这个张量积有一个单位元:即$1$-胞元中的恒等元。事实证明,在任何$\cat{2}$-范畴中,同态范畴$\cat{C}(a, a)$自动成为一个幺半范畴(monoidal category),其中张量积由$1$-胞元的组合定义。结合律(associativity)和单位律(unit laws)简单地由相应的范畴法则推导出来。

让我们看看这在我们的标准$\cat{2}$-范畴$\Cat$的例子中意味着什么。$\Cat(a, a)$的同态范畴是关于$A$的自函子(endofunctor)范畴,其中自函子组合作为它的张量积,恒等函子是与此张量积相关的单位元。我们之前已经看到,自函子形成一个幺半范畴(我们在定义单子(monad)时使用了这个事实),但现在我们看到,这是一个更普遍的现象:任何$\cat{2}$-范畴中的endo-$1$-胞元形成一个幺半范畴。稍后我们会回到这一点,当我们推广单子时。

你可能还记得,在一般的幺半范畴中,我们没有坚持必须严格满足幺半群律(monoid laws)。通常来说,只需在自然同构(isomorphism)下满足结合律(associativity)和单位律(unit laws)就可以了。在$\cat{2}$-范畴中,$\cat{C}(a, a)$中的幺半群律直接由$1$-胞元的组合法则推导出来。这些法则是严格的,所以我们将始终得到一个严格的幺半范畴。然而,也可以放宽这些法则。我们可以说,例如,恒等$1$-胞元$\idarrow[a]$与另一个$1$-胞元$f \Colon a \to b$的组合是同构的,而不是相等的。同构的$1$-胞元通过$2$-胞元定义。换句话说,有一个$2$-胞元:

$$\rho \Colon f \circ \idarrow[a] \to f$$

该$2$-胞元具有逆元。

\begin{figure}[H]
  \centering
  \includegraphics[width=0.35\textwidth]{images/bicat.png}
  \caption{在双范畴中的恒等律在同构(一个可逆的$2$-胞元$\rho$)下成立。}
\end{figure}

\noindent
我们可以对左恒等律(left identity)和结合律(associativity laws)做同样的处理。这种放宽后的$\cat{2}$-范畴称为双范畴(bicategory)(还有一些附加的相干律,这里我将省略)。

正如预期的那样,双范畴中的endo-$1$-胞元形成一个一般的幺半范畴,具有非严格的法则。

双范畴的一个有趣例子是跨范畴(category of spans)。两个对象$a$和$b$之间的跨(span)是一个对象$x$和一对态射:

\begin{gather*}
  f \Colon x \to a \\
  g \Colon x \to b
\end{gather*}

\begin{figure}[H]
  \centering
  \includegraphics[width=0.35\textwidth]{images/span.png}
\end{figure}

\noindent
你可能还记得我们在定义笛卡尔积(categorical product)时使用了跨。在这里,我们希望将跨视为双范畴中的$1$-胞元。第一步是定义跨的组合。假设我们有一个相邻的跨:

\begin{gather*}
  f' \Colon y \to b \\
  g' \Colon y \to c
\end{gather*}

\begin{figure}[H]
  \centering
  \includegraphics[width=0.5\textwidth]{images/compspan.png}
\end{figure}

\noindent
组合将是第三个跨,顶点是某个$z$。最自然的选择是$f'$沿$g$的拉回(pullback)。记住,拉回是对象$z$以及两个态射:

\begin{align*}
  h  & \Colon z \to x \\
  h' & \Colon z \to y
\end{align*}

使得:

$$g \circ h = f' \circ h'$$

并且在所有这样的对象中是通用的。

\begin{figure}[H]
  \centering
  \includegraphics[width=0.5\textwidth]{images/pullspan.png}
\end{figure}

\noindent
目前,我们集中讨论集合范畴中的跨。在这种情况下,拉回只是笛卡尔积$x \times y$中的对的集合$(p, q)$,使得:

$$g\ p = f'\ q$$

两个共享相同终点的跨之间的态射被定义为其顶点之间的一个态射$h$,使得适当的三角形交换。

\begin{figure}[H]
  \centering
  \includegraphics[width=0.4\textwidth]{images/morphspan.png}
  \caption{$\cat{Span}$中的$2$-胞元。}
\end{figure}

\noindent
总结一下,在双范畴$\cat{Span}$中:$0$-胞元是集合,$1$-胞元是跨,$2$-胞元是跨的态射。一个恒等$1$-胞元是一个退化的跨,其中所有三个对象都是相同的,两个态射是恒等态射。

我们之前还看到了双范畴的另一个例子:$\cat{Prof}$双范畴,其中$0$-胞元是范畴,$1$-胞元是函子,$2$-胞元是自然变换。函子的组合由余积(coend)给出。

\section{单子(Monads)}

到目前为止,你应该对单子作为自函子范畴中的幺半群的定义非常熟悉了。让我们用新的理解重新审视这个定义,即自函子范畴只是双范畴$\Cat$中的一个小的endo-$1$-胞元范畴。我们知道它是一个幺半范畴:张量积来自于自函子的组合。幺半群被定义为幺半范畴中的一个对象——在这里它将是一个自函子$T$——连同两个态射。自函子之间的态射是自然变换。一个态射将幺半元,即恒等自函子,映射到$T$:

$$\eta \Colon I \to T$$

第二个态射将$T \otimes T$的张量积映射到$T$。张量积由自函子组合给出,所以我们得到:

$$\mu \Colon T \circ T \to T$$

\begin{figure}[H]
  \centering
  \includegraphics[width=0.3\textwidth]{images/monad.png}
\end{figure}

\noindent
我们将这些视为定义单子的两个操作(在Haskell中,它们被称为\code{return}和\code{join}),我们知道幺半群律转化为单子律。

现在让我们从这个定义中删除所有提到自函子的内容。我们从一个双范畴$\cat{C}$开始,并选择一个$0$-胞元$a$。如前所述,同态范畴$\cat{C}(a, a)$是一个幺半范畴。因此,我们可以通过选择一个$1$-胞元$T$和两个$2$-胞元来定义$\cat{C}(a, a)$中的幺半群:

\begin{align*}
  \eta & \Colon I \to T \\
  \mu  & \Colon T \circ T \to T
\end{align*}

满足幺半群律。我们称之为单子。

\begin{figure}[H]
  \centering
  \includegraphics[width=0.3\textwidth]{images/bimonad.png}
\end{figure}

\noindent
这是一个更广义的单子的定义,仅使用$0$-胞元、$1$-胞元和$2$-胞元。在应用于双范畴$\Cat$时,它将归结为通常的单子。但是让我们看看在其他双范畴中会发生什么。

让我们在$\cat{Span}$中构造一个单子。我们选择一个$0$-胞元,它是一个集合,出于将要解释的原因,我将其称为$\mathit{Ob}$。接下来,我们选择一个endo-$1$-胞元:从$\mathit{Ob}$返回到$\mathit{Ob}$的一个跨。它在顶点处有一个集合,我将其称为$\mathit{Ar}$,它配备了两个函数:

\begin{align*}
  \mathit{dom} & \Colon \mathit{Ar} \to \mathit{Ob} \\
  \mathit{cod} & \Colon \mathit{Ar} \to \mathit{Ob}
\end{align*}

\begin{figure}[H]
  \centering
  \includegraphics[width=0.3\textwidth]{images/spanmonad.png}
\end{figure}

\noindent
让我们称$\mathit{Ar}$集合中的元素为“箭头”。如果我还告诉你称$\mathit{Ob}$集合中的元素为“对象”,你可能会明白这指向哪里。两个函数$\mathit{dom}$和$\mathit{cod}$为每个“箭头”分配了一个“对象”的域和余域。

为了使我们的跨成为一个单子,我们需要两个$2$-胞元$\eta$和$\mu$。在这种情况下,幺半元是一个从$\mathit{Ob}$到$\mathit{Ob}$的平凡跨,其顶点位于$\mathit{Ob}$,两个态射是恒等态射。$2$-胞元$\eta$是$\mathit{Ob}$和$\mathit{Ar}$之间的一个函数。换句话说,$\eta$为每个“对象”分配了一个“箭头”。$\cat{Span}$中的$2$-胞元必须满足交换条件——在这种情况下:

\begin{align*}
  \mathit{dom} & \circ \eta = \id \\
  \mathit{cod} & \circ \eta = \id
\end{align*}

\begin{figure}[H]
  \centering
  \includegraphics[width=0.4\textwidth]{images/spanunit.png}
\end{figure}

\noindent
在组件中,这变为:

$$\mathit{dom}\ (\eta\ \mathit{ob}) = \mathit{ob} = \mathit{cod}\ (\eta\ \mathit{ob})$$

其中$\mathit{ob}$是$\mathit{Ob}$中的一个“对象”。换句话说,$\eta$为每个“对象”分配了一个“箭头”,其域和余域都是该“对象”。我们称这个特殊的“箭头”为“恒等箭头”。

第二个$2$-胞元$\mu$作用于跨$\mathit{Ar}$与自身的组合。组合定义为拉回(pullback),因此其元素是$\mathit{Ar}$中的一对箭头$(a_1, a_2)$,其中满足拉回条件:

$$\mathit{cod}\ a_1 = \mathit{dom}\ a_2$$

我们说$a_1$和$a_2$是“可组合的”,因为一个的余域是另一个的域。

\begin{figure}[H]
  \centering
  \includegraphics[width=0.5\textwidth]{images/spanmul.png}
\end{figure}

\noindent
$2$-胞元$\mu$是一个函数,它将一对可组合的箭头$(a_1, a_2)$映射到一个箭头$a_3$。换句话说,$\mu$定义了箭头的组合。

很容易验证单子律对应于箭头的恒等律和结合律。我们刚刚定义了一个范畴(一个小范畴,注意,其中对象和箭头构成集合)。

因此,总而言之,一个范畴只是一个在跨双范畴中的单子。

这项工作的令人惊奇之处在于,它将范畴置于与单子和幺半群等其他代数结构相同的地位。成为一个范畴并没有什么特别之处。它只是两个集合和几个函数。事实上,我们甚至不需要为对象单独设立一个集合,因为对象可以与恒等箭头同一视角(它们是一一对应的)。所以它实际上只是一个集合和一些函数。考虑到范畴论在整个数学中的核心作用,这是一个非常令人谦卑的认识。

\section{挑战(Challenges)}

\begin{enumerate}
  \tightlist
  \item
  推导双范畴中endo-$1$-胞元组合定义的张量积的单位律和结合律。
  \item
  检查双范畴中一个单子的单子律是否对应于生成范畴中的恒等律和结合律。
  \item
  展示在$\cat{Prof}$双范畴中的一个单子是一个对象上的恒等函子。
  \item
  在$\cat{Span}$双范畴中的单子的单子代数(monad algebra)是什么?
\end{enumerate}

\section{参考文献(Bibliography)}
\begin{enumerate}
  \tightlist
  \item
  \urlref{https://graphicallinearalgebra.net/2017/04/16/a-monoid-is-a-category-a-category-is-a-monad-a-monad-is-a-monoid/}{Paweł Sobociński’s blog}.
\end{enumerate}
